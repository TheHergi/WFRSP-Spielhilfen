\documentclass[a4paper, fontsize=11pt]{scrartcl}

\usepackage{multirow}
\usepackage[margin=20mm]{geometry}
\usepackage[graphicx]{realboxes}
\usepackage{tabularx}
\usepackage[table,xcdraw]{xcolor}
\usepackage{times}
\usepackage[labelformat=empty]{caption} % remove caption
\usepackage{multicol}
%\renewcommand{\familydefault}{\sfdefault}
\renewcommand{\arraystretch}{1.2}
\newcolumntype{R}{>{\raggedright\arraybackslash}X}%


\begin{document}

%\pagestyle{empty} % remove page numbers
\begin{table}[!ht]
    \centering
    \begin{tabularx}{\textwidth}{l|l|l}
        \multicolumn{3}{c}{\cellcolor{gray!25} \textbf{EP-Kosten beim Verbessern von Werten und Fähigkeiten}                 } \\ \cline{1-3}
        \multirow{2}{*}{Steigerungen} & \multicolumn{2}{c}{EP-Kosten pro Steigerung}                                           \\ \cline{2-3}
                                      & Spielwert                                    & Fähigkeit                               \\ \cline{1-3}
        0 bis 5                       & 25                                           & 10                                      \\ \cline{1-3}
        6 bis 10                      & 30                                           & 15                                      \\ \cline{1-3}
        11 bis 15                     & 40                                           & 20                                      \\ \cline{1-3}
        16 bis 20                     & 50                                           & 30                                      \\ \cline{1-3}
        21 bis 25                     & 70                                           & 40                                      \\ \cline{1-3}
        26 bis 30                     & 90                                           & 60                                      \\ \cline{1-3}
        31 bis 35                     & 120                                          & 80                                      \\ \cline{1-3}
        36 bis 40                     & 150                                          & 110                                     \\ \cline{1-3}
        41 bis 45                     & 190                                          & 140                                     \\ \cline{1-3}
        46 bis 50                     & 230                                          & 180                                     \\ \cline{1-3}
        51 bis 55                     & 280                                          & 220                                     \\ \cline{1-3}
        56 bis 60                     & 330                                          & 270                                     \\ \cline{1-3}
        61 bis 65                     & 390                                          & 320                                     \\ \cline{1-3}
        66 bis 70                     & 450                                          & 380                                     \\ \cline{1-3}
        70+                           & 520                                          & 440
    \end{tabularx}
\end{table}

\begin{table}[!ht]
    \centering
    \begin{tabularx}{\textwidth}{l|X}
        \multicolumn{2}{c}{\cellcolor{gray!25} \textbf{EP-Kosten von Talenten und Karrierewechseln}                                                 } \\ \cline{1-2}
        \textbf{Verbesserung}                    & \textbf{EP-Kosten}                                                                                 \\ \cline{1-2}
        +1 Talent                                & 100 EP + 100 EP für des Mal, dass dieses Talent bereits gelernt wurde                              \\ \cline{1-2}
        Eine abgeschlossene Karriere verlassen   & 100 EP                                                                                             \\ \cline{1-2}
        Eine unabgeschlossene Karriere verlassen & 200 EP                                                                                             \\ \cline{1-2}
        In eine andere Klasse wechseln           & +100 EP
    \end{tabularx}
\end{table}

\begin{table}[!ht]
    \centering
    \begin{tabularx}{\textwidth}{l|X}
        \multicolumn{2}{c}{\cellcolor{gray!25} \textbf{Eine Karriere abschließen}                                                 } \\ \cline{1-2}
        \textbf{Stufe} & \textbf{Voraussetzungen}                                                                                   \\ \cline{1-2}
        1              & 5 Steigerungen + 1 Talent                                                                                  \\ \cline{1-2}
        2              & 10 Steigerungen+ 1 Talent                                                                                  \\ \cline{1-2}
        3              & 15 Steigerungen+ 1 Talent                                                                                  \\ \cline{1-2}
        4              & 20 Steigerungen + 1 Talent
    \end{tabularx}
    \caption{
        Die Steigerungen sind dabei in allen der Karrierestufe verfügbaren Spielwerten, und in 8 Fähigkeiten vorzunehmen.
    }
\end{table}

\end{document}