\documentclass{article}

\usepackage[landscape, margin=1cm, a4paper]{geometry}
\usepackage{multicol}
\usepackage{multirow}
\usepackage{tikz}
\usepackage{xcolor}
\usepackage{enumitem}
\usepackage{blindtext}
\usepackage{tabularx}
\usepackage{environ}
\usepackage{tcolorbox}
\usepackage{titlesec}

\usepackage[german]{babel}
\usepackage[utf8]{inputenc}

% \usepackage{showframe}

\setlength{\columnsep}{-0.2cm}

\NewEnviron{slsframe}[2][0.85\linewidth]
{
    \begin{tikzpicture}
        \hspace{-0.25cm}
        \node [draw=black, fill=white, very thick, rectangle, rounded corners, inner xsep=0.5em, inner ysep=0.5em] (box) 
                {
                    % \noindent
                    
                    \begin{minipage}[t]{#1}
                        
                    % \noindent
                    % \begin{flushleft}
                    \BODY
                    % \end{flushleft}
                    \end{minipage}
                    };
        \node[fill=black, text=white, font=\bfseries, yshift=0.68em] at (box.north) {#2};
    \end{tikzpicture}
}
\renewcommand\tabularxcolumn[1]{m{#1}}% for vertical centering text in X column


\setlist[itemize]{noitemsep, topsep=0pt}
\titlespacing\subsubsection{0pt}{8pt plus 4pt minus 2pt}{0pt plus 2pt minus 2pt}
% \setlist[itemize]{itemsep=0.1em, topsep=0pt}

\begin{document}


\begin{multicols*}{3}
    
    \begin{slsframe}{Ups-Tabelle}        
            \begin{tabularx}{\linewidth}{lX}
            % \begin{tabularx}{\linewidth}{@{}>{\bfseries}l@{\hspace{.5em}}X@{}}
                01-20 & Du triffst einen Teil der eigenen Anatomie (wir empfehlen, dies auf amüsante Weise auszuspielen). Du verlierst 1 LP, wobei Widerstands-Bonus und Rüstungspunkte ignoriert werden.  \\ \hline

                21-40 & Deine Nahkampfwaffe verkantet sich unglücklich oder deine Fernkampfwaffe hat eine Fehlfunktion und irgendetwas bricht ab. Deine Waffe erleidet 1 Punkt Schaden. In der nächsten Runde bist du als Letzter dran, unabhängig von der Initiative-Reihenfolge, Talenten oder Sonderregeln, da du dich erholen musst (siehe Seite 156).\\ \hline

                41-60 & Dein Manöver lief ins Leere und bringt dich aus dem Tritt oder du verlierst den Halt an deiner Fernkampfwaffe. In der nächsten Runde hast du bei deiner Handlung einen Abzug von –10.\\ \hline

                61-70 & Du stolperst und brauchst einen Moment, um wieder richtig in Position zu kommen. Du verlierst deine nächste Bewegungsaktion.  \\ \hline

                71-80 & Du greifst deine Waffe falsch oder verlierst deine Munition. Du verlierst deine nächste Handlung.  \\ \hline

                81-90 & Du überstreckst dich oder stolperst und verdrehst deinen Knöchel. Du erleidest eine Muskelriss (leicht)-Verletzung (siehe Seite 179). Dies gilt als Kritische Verletzung. \\ \hline

                91-00 & Du verhaust es komplett und triffst einen zufälligen Verbündeten in Reichweite, wobei dein geworfener Würfel für die Einerstelle über die EG des Treffers entscheidet. Wenn das nicht möglich ist, schlägst du dir irgendwie selbst ins Gesicht und erhältst einen Betäubt-Zustand (siehe Seite 167).

            \end{tabularx}        
    \end{slsframe}
    

    \begin{slsframe}[1.3\linewidth]{Fernkampfattacke}
        \begin{tabularx}{\linewidth}{@{}>{\bfseries}l@{\hspace{.5em}}X@{}}
            \multirow{2}{*}{+60}    & ... auf ein Monströses Ziel (Riesen-Größe)                                                       \\
                                    & ... in eine Menge (13+ Ziele)                                                                    \\ \hline
            \multirow{4}{*}{+40}    & ... auf ein Ziel in Kernschussweite (Reichweite / 10)                                            \\
                                    & ... auf ein enormes Ziel (Greifen-Größe)                                                         \\
                                    & ... auf eine große Gruppe (7-12 Ziele)                                                           \\ \hline
            \multirow{4}{*}{+20}    & ... auf ein großes Ziel (Oger-Größe)                                                             \\
                                    & ... bei kurzer Reichweite (weniger als halbe Reichweite der Waffe)                               \\
                                    & ... auf eine kleine Gruppe (3-6 Ziele)                                                           \\
                                    & ... bei der es deine letzte Handlung war, zu zielen  \\ \hline
            +0                      & ... auf ein normales Ziel (Menschen-Größe)                                                       \\ \hline
            \multirow{4}{*}{-10}    & ... bei langer Reichweite (bis zu doppelter Reichweite der Waffe)                                \\
                                    & ... in einer Runde, in der du auch eine Bewegungsaktion durchführst                              \\
                                    & ... auf ein zierliches Ziel (Halbling-Größe)                                                     \\
                                    & ... auf Gegner in leichter Deckung (zum Beispiel hinter einer Hecke)                                                  \\ \hline
            \multirow{4}{*}{-20}    & ... auf eine bestimmte Trefferzone           \\
                                    & ... auf Ziele, die von Nebel oder Dunkelheit verborgen werden                                    \\
                                    & ... auf ein kleines Ziel (Katzen-Größe)                                                          \\
                                    & ... auf Gegner in mittlerer Deckung (zum Beispiel hinter einem Holzzaun)                                              \\ \hline
            \multirow{4}{*}{-30}    & ... auf ein winziges Ziel (Mäuse-Größe)                                                          \\
                                    & ... bei extremer Reichweite (bis zur dreifachen Reichweite der Waffe)                            \\
                                    & ... bei Dunkelheit                                                                               \\
                                    & ... auf Gegner in schwerer Deckung (zum Beispiel hinter einer Steinmauer)                                                                    
        \end{tabularx}
    
    \end{slsframe}

    \begin{slsframe}[1.3\linewidth]{Nahkampfattacke}
        \begin{tabularx}{\linewidth}{@{}>{\bfseries}l@{\hspace{.5em}}X@{}}
            +40                  & ... auf einen Gegner, bei dem man 3 zu 1 in der Überzahl ist                                           \\ \hline
            \multirow{3}{*}{+20} & ... in den Rücken oder die Seite eines Gegners, der Gebunden ist                                       \\
                                 & ... auf einen Gegner, bei dem man 2 zu 1 in der Überzahl ist                                           \\
                                 & ... auf einen Gegner, der den Zustand Niedergestreckt hat                                              \\ \hline
            +0                   & ... Standard                                                                                               \\ \hline
            \multirow{2}{*}{-10} & ... während man den Zustand Niedergestreckt hat                                                       \\
                                 & ... inmitten von Schlamm, dichtem Regen oder schwierigem Gelände                                       \\ \hline
            \multirow{6}{*}{-20} & ... auf eine bestimmte Trefferzone                \\
                                 & ... in beengtem Raum mit einer überdurchschnittlich langen Waffe     \\
                                 & ... inmitten eines Sturzregens, Schneesturms oder Extremwetters anderer Art                    \\
                                 & ... oder Ausweichen, während man den Zustand Niedergestreckt hat                                                            \\
                                 & ... bei Dunkelheit                                                                                            \\
                                 & ... mit einer Waffe in der nichtdominanten Hand                                                                   \\ \hline
            -30                  & ... oder Ausweichen in tiefem Schnee, Wasser oder anderer, die Bewegung stark einschränkender Umgebung
        \end{tabularx}    
    \end{slsframe}

    \hspace{4cm}
    \begin{slsframe}[0.45\linewidth]{Trefferzonen}
        \begin{tabularx}{\linewidth}{@{}>{\bfseries}l@{\hspace{.5em}}X@{}}
            01-09 & Kopf \\ 
            10-24 & Linker Arm \\ 
            25-44 & Rechter Arm \\ 
            45-79 & Körper \\ 
            80-89 & Linker Fuß \\ 
            90-00 & Rechter Fuß \\
        \end{tabularx}
    \end{slsframe}

    \hspace{4cm}
    \begin{slsframe}[0.45\linewidth]{Größe}
        \begin{tabularx}{\linewidth}{Xrr}
            \textbf{Größe}    & \textbf{Höhe}   & \textbf{Mod.} \\
            Winzig   & $<30cm$  & -30  \\
            Klein    & $<60cm$  & -20  \\
            Zierlich & $<1,2m$  & -10  \\
            Normal   & $<2,1m$  & 0    \\
            Groß     & $<4m  $  & +20  \\
            Enorm    & $<6m  $  & +40  \\
            Monströs & $>6m $ & +60 
        \end{tabularx}
    \end{slsframe}

    % \hspace{4cm}
    % \begin{slsframe}[0.45\linewidth]{Größenmodifikationen}
    %     \begin{itemize}
    %         \item Wenn die Kreatur 1 Stufe größer ist, erhalten ihre Waffen die Qualität Verwundend; wenn die Kreatur 2 oder mehr Stufe größer ist, erhalten ihre Waffen die Qualität Wuchtig.
    %     \end{itemize}
    % \end{slsframe}

\end{multicols*}

\pagebreak

\begin{multicols*}{3}

    \begin{slsframe}{Geld}
        \begin{tabularx}{\linewidth}{Xr}
            1 Goldkrone & 20 Silberschillinge   \\
            1 Goldkrone    & 240 Messinggroschen   \\
            1 Silberschilling    & 12 Messinggroschen   
        \end{tabularx}
    \end{slsframe}

    \begin{slsframe}{Speisen \& Getränke}
        \begin{tabularx}{0.95\linewidth}{Xl}
            Bier (Humpen)                       & 3 G            \\ 
            Bierfass                            & 3 S            \\ 
            Bugmans XXXXXX                      & 9 G            \\ 
            Essen im Gasthaus                   & 1 S            \\ 
            Gemeinschaftsunterkunft (pro Nacht) & 10 G           \\ 
            Lebensmittel (pro Tag)              & 10 G           \\ 
            Rationen (pro Tag)                  & 2 S            \\ 
            Schnaps (Krug)                      & 2 S            \\ 
            Stall (pro Nacht)                   & 10 G           \\ 
            Wein \& Schnaps (Becher)            & 1 S            \\ 
            Wein (Flasche)                      & 10 G           \\ 
            Zimmer (pro Nacht)                  & 10 S               
        \end{tabularx}%
    \end{slsframe}

    \begin{slsframe}{Reisekosten}
        \begin{tabularx}{\linewidth}{Xrrl}
            \textbf{Transport} & \textbf{M} & \textbf{Kosten} & \textbf{Distanz} \\
            Kutsche (innen) & 6 & 2 G & pro IM  \\
            Kutsche (außen) & 6 & 1 G & pro IM  \\
            Barke (Kabine) & 8  & 5 G & pro IM  \\
            Barke (Deck) & 8    & 2 G & pro IM  \\
            Droschke & 6        & 3 G & pro Viertel  \\
            Fähre & 4           & 1 G & pro $20m$  
        \end{tabularx}
    \end{slsframe}


    \begin{slsframe}{Fähigkeiten}
        \begin{tabularx}{\linewidth}{llll}
            \multicolumn{2}{c}{\textbf{Grundfähigkeiten}} & \multicolumn{2}{c}{\textbf{Ausbaufähigkeiten}} \\
            Anführen                & CH & Abrichten \ldots & IN\\
            Athletik                & GW & Artistik \ldots & GW\\
            Ausdauer                & WI & Beruf \ldots & GS\\
            Ausweichen              & GW & Beten & CH \\
            Besonnenheit            & WK & Fallen stellen & GS \\
            Bestechen               & CH & Fernkampf \ldots & BF\\
            Charme                  & CH & Geheimzeichen \ldots & IN\\
            Einschüchtern           & ST & Heilen & IN \\
            Fahren                  & GW & Kanalisieren \ldots & WK  \\
            Feilschen               & CH & Kunst \ldots & GS\\
            Fingerfertigkeit        & GS & Musizieren \ldots & GS\\
            Glücksspiel             & IN & Nachforschen & IN \\
            Intuition               & I  & Schätzen & IN \\
            Klatsch                 & CH & Schlösser öffnen & GS \\
            Klettern                & ST & Schwimmen & ST \\
            Nahkampf \ldots         & KG & Segeln \ldots & GW\\
            Navigation              & I  & Sprache \ldots & IN\\
            Reiten \ldots           & GW & Spurenlesen & I \\
            Rudern                  & ST & Tierpflege & IN \\
            Schleichen \ldots       & GW & Wissen \ldots & IN  \\
            Tiere bezirzen          & WK &                       &  \\
            Überleben               & IN &                       &  \\
            Unterhalten \ldots      & CH &                       &  \\
            Wahrnehmung             & I  &                       &   \\
            Zechen                  & WI &                       &  
        
        \end{tabularx}
        \end{slsframe}


\end{multicols*}

\pagebreak

\begin{multicols*}{3}
    
    \begin{slsframe}{Größenmodifikationen als Angreifer}
        \begin{tabularx}{\linewidth}{rX}
            \textbf{$\Delta$}       & \textbf{Effekt} \\ \hline
            $<0$                    & +10 auf Treffer \\ \hline
            \multirow{2}{*}{$=1$}   & Qualität \textit{Verwundend} \\ 
                                    & Verursacht \textit{Angst}(1) \\ \hline
            \multirow{3}{*}{$>1$}   & Qualität \textit{Wuchtig}. \\
                                    & Schaden wird mit $\Delta$ multipliziert.  \\
                                    & Verursacht \textit{Angst}(2) \\ \hline
            $>2$                    & Verursacht \textit{Entsetzen} ($\Delta$)
        \end{tabularx}

        
        \begin{itemize}
            \item Verteidigung: Für jede Stufe, die dein Gegner größer ist, erhältst du -2 EG auf parieren.
        \end{itemize}

    \end{slsframe}

    \begin{slsframe}{Psychologie}

        \subsubsection*{Angst}
        \begin{itemize}
            \setlength\itemsep{0.1em}
            \item SC muss bei einem erweiterten \textit{Besonnenheits}-Wurf $\Delta$ EG erreichen.
            \item Gewürfelt wird am Ende jeder Runde.
            \item -10 bei allen Würfen gegen die Quelle dieser Angst.
            \item Näherung mit \textit{Besonnenheits}-Wurf.
            \item Auf einem zukommt: \textit{Besonnenheits}-Wurf, sonst \textit{Demoralisiert}.  
        \end{itemize}
        
        \subsubsection*{Entsetzen}
        \begin{itemize}
            \setlength\itemsep{0.1em}
            \item \textit{Besonnenheits}-Wurf, sonst $\Delta$ \textit{Demoralisiert}-Zustände. 
            \item Kreatur verursacht \textit{Angst} ($\Delta$).
        \end{itemize}

    \end{slsframe}

    \begin{slsframe}{Gesundheit}
        \subsubsection*{Verwundung}

    \end{slsframe}

    \begin{slsframe}{Qualitäten}
        \subsubsection*{Durchbohrend}
        Eine durchbohrende Waffe verursacht bei jeder Zahl, die durch 10 teilbar ist sowie bei jedem Pasch einen Kritischen Treffer.

        \subsubsection*{Rüstungsbrechend}
        RP durch nichtmetallische Rüstteile werden ignoriert und bei jeder anderen Rüstung wird der erste Punkt ignoriert.

        \subsubsection*{Schießpulver}
        Wenn du das Ziel einer Schießpulver-Waffe bist, dann musst du einen (+20) \textit{Besonnenheits}-Wurf bestehen, oder du erhältst den Zustand Demoralisiert, selbst wenn der Schuss verfehlt.

        \subsubsection*{Verwundend}
        Eine verwundende Waffe kann zur Bestimmung des Schadens entweder die Einerstelle des Würfelwurfes oder die EG verwenden werden.
        
        \subsubsection*{Wuchtig}
        Wenn du einen Gegner triffst, addierst du das Resultat des Würfels für die Einerstelle zu jedem Schaden hinzu.

    \end{slsframe}

    \begin{slsframe}{Makel}
        \subsubsection*{Gefährlich}
        Jeder fehlgeschlagene Wurf, bei dem der Würfel für die Einerstelle 9 oder eine 10 zeigt, führt zu einem Patzer.

        \subsubsection*{Nachladen (Wert)}
        Eine nicht geladene Waffe mit diesem Makel zu laden erfordert einen Erweiterten Fernkampf-Wurf für die jeweilige Waffengruppe, der (Wert) EG benötigt.

    \end{slsframe}



    % \begin{slsframe}{Zustände}
    %     \subsubsection*{Betäubt}
    %     % \textbf{Betäubt}
    %     \begin{itemize}
    %         \setlength\itemsep{-0.2em}
    %         \item Keine Handlung (nur Bewegung mit Hälfte)
    %         \item -10 auf alle Würfe (Gegner +1 SL)
    %         \item Ende jeder Runde \textit{Ausdauer}-Wurf, wobei EG-Betäubt-Zustand entfernt werden.
    %         \item Wenn alle Betäubt-Zustände entfernt wurden, erhältst du Erschöpft-Zustand.
    %     \end{itemize}

    %     \subsubsection*{Blutend}
    %     \begin{itemize}
    %         \setlength\itemsep{-0.2em}
    %         \item Ende jeder Runde -1 LP (bis zu 0 LP; unabhängig von Modifikatoren)
    %         \item Bei 0 LP: Bewusstlos, dann pro Blutend-Zustand eine Chance von 10\% zu sterben. Bei Pasch: -1 Blutend-Zustand
    %         \item Bewusstsein nur wenn alle Blutend-Zustände entfernt wurden, dann Erschöpft-Zustand
    %         \item \textit{Heilen}: $1 + EG$-\textit{Heilen} Blutend-Zustand entfernt
    %     \end{itemize}        

    %     \subsubsection*{Erschöpft}
    %     \begin{itemize}
    %         \setlength\itemsep{-0.2em}
    %         \item -10 auf alle Würfe
    %     \end{itemize}        

    %     \subsubsection*{Überrascht}
    %     \begin{itemize}
    %         \setlength\itemsep{-0.2em}
    %         \item Keine (Bewegung-)Aktion 
    %         \item Bei Vergleichenden Würfen nicht verteidigen
    %         \item +20 auf Nahkampf-Trefferwurf.
    %     \end{itemize}        

    %     \subsubsection*{Niedergestreckt}
    %     \begin{itemize}
    %         \setlength\itemsep{-0.2em}
    %         \item Bewegungsaktion nur zum Aufstehen oder kriechen 
    %         \item Mit 0 LP nur noch kriechen (Hälfte deines Bewegungs-Wertes in Metern)
    %         \item -20 auf Bewegungsabhängige Würfe.
    %         \item Gegner erhält +20 auf Nahkampf-Trefferwurf

    %     \end{itemize}    
        

    %     \subsubsection*{Demoralisiert}
    % \end{slsframe}

    \begin{slsframe}{Zustände}
    % \begin{slsframe}{Erschöpft-Zustand}
        \subsubsection*{Erschöpft}
        \begin{itemize}%[noitemsep,topsep=0pt]
            \item -10 auf alle Würfe
        \end{itemize}        
    % \end{slsframe}

    % \begin{slsframe}{Betäubt-Zustand}
        \subsubsection*{Betäubt}
        \begin{itemize}%[noitemsep,topsep=0pt]
            \item Keine Handlung (nur Bewegung mit Hälfte)
            \item -10 auf alle Würfe (Gegner +1 SL)
            \item Ende jeder Runde \textit{Ausdauer}-Wurf, wobei EG-Betäubt-Zustand entfernt werden.
            \item Wenn alle Betäubt-Zustände entfernt wurden, erhältst du Erschöpft-Zustand.
        \end{itemize}
    % \end{slsframe}

    % \begin{slsframe}{Blutend-Zustand}
    \subsubsection*{Blutend}
        \begin{itemize}%[noitemsep,topsep=1em]
            \item Ende jeder Runde -1 LP (bis zu 0 LP; unabhängig von Modifikatoren)
            \item Bei 0 LP: Bewusstlos, dann pro Blutend-Zustand eine Chance von 10\% zu sterben. Bei Pasch: -1 Blutend-Zustand
            \item Bewusstsein nur wenn alle Blutend-Zustände entfernt wurden, dann Erschöpft-Zustand
            \item \textit{Heilen} entfernt $1 + EG$-\textit{Heilen} Zustände
        \end{itemize}        
    % \end{slsframe}

    % \begin{slsframe}{Überrascht-Zustand}
    \subsubsection*{Überrascht}
        \begin{itemize}%[noitemsep,topsep=0pt]
            \item Keine (Bewegung-)Aktion 
            \item Bei Vergleichenden Würfen nicht verteidigen
            \item +20 auf Nahkampf-Trefferwurf.
        \end{itemize}   
    % \end{slsframe}     

    % \begin{slsframe}{Niedergestreckt-Zustand}
    \subsubsection*{Niedergestreckt}
        \begin{itemize}%[noitemsep,topsep=0pt]
            \item Bewegungsaktion: Aufstehen oder kriechen 
            \item Mit 0 LP nur noch kriechen (Hälfte deines Bewegungs-Wertes in Metern)
            \item -20 auf Bewegungsabhängige Würfe.
            \item Gegner erhält +20 auf Nahkampf-Trefferwurf
        \end{itemize}   
    % \end{slsframe} 
        

    % \begin{slsframe}{Demoralisiert-Zustand}
    \subsubsection*{Demoralisiert}
        \begin{itemize}%[noitemsep,topsep=0pt]
            \item Muss (Bewegungs-) Aktion nutzen, um sich zu verstecken oder fliehen.
            \item -10 auf alle Würfe außer fliehen/verstecken.
            \item Nicht überwindbar wenn gebunden, sonst reduziert Besonnenheits-Wurf pro EG.
            \item Wenn eine komplette Runde verbracht wurde, sich außer Sicht jedes Feindes zu verstecken, $-1$ Demoralisiert-Zustand.
            \item Sobald alle Demoralisiert-Zustände entfernt wurden: 1 Erschöpft-Zustand.
        \end{itemize}   
    % \end{slsframe}
\end{slsframe}
        

\end{multicols*}

\end{document}