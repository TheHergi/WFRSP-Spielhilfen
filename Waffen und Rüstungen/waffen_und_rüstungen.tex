\documentclass[a4paper, fontsize=9pt twocolumn]{scrartcl}

\usepackage[margin=20mm]{geometry}
\usepackage{times}
\usepackage[labelformat=empty]{caption} % remove caption
\usepackage{multicol}

\begin{document}
\section{Waffen}
\begin{multicols*}{2}


    \subsection{Ablenkend}
    Ablenkende Waffen sind gefährlich oder peitschenartig und können verwendet werden, um einen Gegner zurückzudrängen.
    Statt Schaden zu verursachen, kann eine erfolgreiche Attacke mit einer Ablenkend-Waffe einen Gegner für jeden EG um 1 Meter zurückdrängen, um den du den Vergleichenden Wurf gewonnen hast.

    \subsection{Akkurat}
    Die Waffe ist so akkurat, dass man leichter mit ihr trifft.
    Du erhältst einen Bonus von +10 bei jedem Schuss mit dieser Waffe.

    \subsection{Betäubend}
    Betäubende Waffen sind besonders gut darin, Feinde besinnungslos zu schlagen.
    Wenn dir mit einer betäubenden Waffe ein Kopftreffer gelingt, kannst du einen Vergleichenden Stärke/Ausdauer-Wurf gegen den getroffenen Gegner durchführen. Wenn du den Wurf gewinnst, erhält der Gegner einen Betäubt-Zustand.

    \subsection{Defensiv}
    Defensive Waffen sind zum Parieren gegnerischer Attacken gedacht.
    Wenn du eine solche Waffe führst, bekommst du auf jeden Vergleichenden Nahkampf-Wurf zur Abwehr einer gegnerischen Attacke +1 EG.

    \subsection{Durchbohrend}
    Durchbohrende Waffen können mit einem einzigen sauberen Treffer töten.
    Eine durchbohrende Waffe verursacht bei jeder Zahl, die durch 10 teilbar ist (also 10, 20, 30 usw.) sowie bei jedem Pasch (also 11, 22, 33 etc.) einen Kritischen Treffer, sofern das Wurfergebnis im Kampf nicht oberhalb der benötigten Zahl liegt.

    Wenn eine durchbohrende Waffe eine Fernkampfwaffe ist, dann hat sich deren Geschoss bei einem Treffer tief im Körper des Zieles verhakt.
    Pfeile und Bolzen können nur mit einem jeweils gelungenen herausfordernden (+0) Heilen-Wurf entfernt werden - zum Entfernen von Kugeln ist sogar Chirurgie erforderlich (siehe Kapitel 4: Fähigkeiten und Talente).
    Für jeden nicht entfernten Pfeil, Bolzen oder Kugel kann 1 LP nicht heilen.

    \subsection{Durchschlagend}
    Durchschlagende Waffen haben schwere Klingen, die sich mit erschreckender Leichtigkeit durch Rüstungen brechen.
    Wenn du einen Gegner triffst, dann richtest du nicht nur Schaden am Ziel an, sondern verursachst auch noch 1 Punkt Schaden an seinem Schild oder einem getroffenen Stück Rüstung.

    \subsection{Ermüdend}
    Diese Waffe einzusetzen ist ermüdend oder ihr richtiger Einsatz erfordert Schwung.
    Du profitierst nur dann von den Qualitäten Verwundend oder Wuchtig dieser Waffe, wenn du in deinem Zug einen Sturmangriff durchgeführt hast.

    \subsection{Fesselnd}
    Deine Waffe wickelt sich um deinen Gegner und fesselt ihn. Jeder Gegner, den du erfolgreich mit deiner Waffe triffst, erhält den Gefesselt-Zustand mit einer Stärke, die deinem Stärke-Wert entspricht.
    Wenn du gerade einen Gegner gefesselt hast, kannst du die Waffe nicht für andere Attacken verwenden.
    Du kannst das Fesseln jederzeit abbrechen.

    \subsection{Gefährlich}
    Bei manchen Waffen ist die Chance, dich selbst zu verletzen fast genauso hoch, wie die, deinen Gegner zu verletzen.
    Jeder fehlgeschlagene Wurf, bei dem der Würfel für die Einerstelle 9 oder eine 10 zeigt, führt zu einem Patzer (weitere Informationen zu Patzern findest du in Kapitel 5: Regeln).

    \subsection{Klingenfänger}
    Einige Waffen sind so konzipiert, dass sie andere Waffen fangen und manchmal sogar brechen können.
    Wenn du einen Kritischen Treffer erzielst, während du dich gegen eine Attacke mit einer Klingenwaffe verteidigst, dann kannst du dich dafür entscheiden, die Klinge zu fangen, statt einen Kritischen Treffer zu verursachen.
    Wenn du dich dafür entscheidest, führst du einen Vergleichenden Stärke-Wurf durch, zu dem die EG aus deinem vorherigen Nahkampf-Wurf dazuzählen.
    Wenn du erfolgreich bist, lässt dein Gegner die Klinge fallen, da sie ihm aus den Händen gehebelt wird.
    Wenn du einen Verblüffenden Erfolg (+6) erzielst, dann entwaffnest du deinen Gegner nicht nur, sondern brichst durch Kraft und Hebelwirkung auch noch dessen Klinge, sofern diese nicht unzerbrechlich ist.
    Wenn du den Wurf nicht gewinnst, befreit der Gegner seine Klinge und ihr kämpft wie üblich weiter.

    \subsection{Langsam}
    Langsame Waffen sind unhandlich und massig, was ihre Handhabung erschwert.
    Charaktere, die langsame Waffen verwenden, schlagen unabhängig von der Initiativereihenfolge immer als Letzte in der Runde zu.
    Außerdem erhalten Gegner einen Bonus von +1 EG bei jedem Wurf, mit dem sie sich gegen Attacken mit einer solchen Waffe verteidigen.

    \subsection{Nachladen (Wert)}
    Die Waffe lässt sich nur langsam nachladen. Eine nicht geladene Waffe mit diesem Makel zu laden erfordert einen Erweiterten Fernkampf-Wurf für die jeweilige Waffengruppe, der (Wert) EG benötigt.
    Wenn du beim Nachladen unterbrochen wirst, musst du noch mal ganz von vorne beginnen.

    \subsection{Pistole}
    Mit dieser Waffe kannst du im Nahkampf attackieren.

    \subsection{Präzise}
    Mit dieser Waffe lässt sich sehr genau treffen. Du erhältst einen Bonus von +1 EG auf jeden gelungenen Wurf bei der Attacke mit dieser Waffe.

    \subsection{Radius (Wert)}
    Alle Charaktere innerhalb von (Wert) Metern vom getroffenen Punkt, erleiden EG+Waffenschaden und bekommen alle Zustände, die die Waffe verursacht.

    \subsection{Repetierend (Wert)}
    Deine Waffe enthält (Wert) Schüsse und lädt nach jedem Schuss automatisch nach.
    Wenn du alle Schüsse verbraucht hast, musst du die Waffe nach den normalen Regeln vollständig nachladen, ehe du erneut schießen kannst.

    \subsection{Rüstungsbrechend}
    Die Waffe ist extrem effektiv beim Durchdringen von Rüstungen.
    RP durch nichtmetallische Rüstteile werden ignoriert und bei jeder anderen Rüstung wird der erste Punkt ignoriert.

    \subsection{Schießpulver}
    Das Donnern einer Pulverwaffe, gefolgt von Rauch und Verwirrung, kann verstörend sein.
    Wenn du das Ziel einer Schießpulver-Waffe bist, dann musst du einen durchschnittlichen (+20) Besonnenheits-Wurf bestehen, oder du erhältst den Zustand Demoralisiert, selbst wenn der Schuss verfehlt.

    \subsection{Schild (Wert)}
    Wenn du diese Waffe verwendest, um einer Attacke zu widerstehen, dann giltst du, als hättest du an allen Trefferzonen zusätzlich (Wert) Rüstungspunkte.
    Wenn deine Waffe einen Schild-Wert von 2 oder höher hat (also Schild 2 oder Schild 3), dann kannst du auch vergleichend gegen Fernkampf-Attacken in deiner Sichtlinie würfeln.

    \subsection{Schnell}
    Schnelle Waffen sind so konstruiert, dass sie mit so hoher Geschwindigkeit nach dem Gegner schlagen oder stechen, dass das Parieren fast unmöglich ist und der Gegner schon getroffen ist, ehe er reagieren konnte.
    Wer eine schnelle Waffe führt, kann sich entscheiden, mit dieser Waffe außerhalb der normalen Initiativereihenfolge zu attackieren.
    Dabei kann er entweder zuerst, zuletzt oder irgendwo zwischendurch attackieren, ganz wie er wünscht.\newline

    Zudem erleiden alle Nahkampf-Würfe zur Abwehr von schnellen Waffen einen Abzug von -10, wenn der Gegner nicht ebenfalls eine Waffe mit der Qualität Schnell verwendet; andere Fähigkeiten verteidigen wie gewohnt.
    Zwei Kombattanten mit schnellen Waffen kämpfen wie üblich in der Reihenfolge der Initiative (relativ zueinander).
    Eine schnelle Waffe kann niemals zugleich langsam sein (Langsam hat Vorrang).

    \subsection{Stumpf}
    Einige Waffen sind nicht sehr gut darin, Rüstungen zu durchdringen.
    Gegen stumpfe Waffen werden alle RP verdoppelt.
    Außerdem verursachen sie bei einem erfolgreichen Trefferwurf im Nahkampf nicht automatisch den Verlust von mindestens 1 LP.

    \subsection{Umwickelnd}
    Umwickelnde Waffen haben normalerweise lange Ketten mit Gewichten am Ende, was es sehr schwierig macht, sie effektiv zu parieren.
    Nahkampf-Würfe gegen Angriffe mit einer umwickelnden Waffe haben einen Abzug von -1 EG, weil parierte Hiebe dennoch um den Schildrand oder die Klinge schlagen.

    \subsection{Unpräzise}
    Unpräzise Waffen sind schwer zu kontrollieren, da sie unhandlich oder schlecht balanciert sind.
    Du erleidest einen Abzug von -1 EG, wenn du die Waffe für eine Attacke verwendest.
    Eine unpräzise Waffe kann niemals präzise sein (Unpräzise hat Vorrang).

    \subsection{Unzerbrechlich}
    Die Waffe ist ganz besonders vorzüglich oder aus besonders festem Material gefertigt.
    Unter nahezu allen Umständen wird diese Waffe nicht zerbrechen, korrodieren oder ihre Schärfe verlieren.

    \subsection{Verwundend}
    Eine verwundende Waffe kann zur Bestimmung des Schadens entweder die Einerstelle des Würfelwurfes oder die EG verwenden, je nachdem, was höher ist.
    Wenn du bei deinem Trefferwurf zum Beispiel eine 34 gewürfelt hast und eine 52 brauchtest, um zu treffen, dann kannst du entweder die EG nehmen, die in diesem Fall 2 sind, oder die Einerstelle des Würfelwurfes, was 4 ist. Eine stumpfe Waffe kann niemals verwundend sein.

    \subsection{Wuchtig}
    Einige Waffen sind einfach riesig oder verursachen aufgrund ihres Gewichts oder ihrer Form schreckliche Wunden.
    Wenn du einen Gegner triffst, addierst du das Resultat des Würfels für die Einerstelle zu jedem Schaden hinzu, den eine wuchtige Waffe verursacht.
    Eine stumpfe Waffe kann niemals zugleich wuchtig sein (Stumpf hat Vorrang).

    \section{Rüstungen}

    \subsection{Flexibel}
    Eine flexible Rüstung kann auf Wunsch unter einer nicht-flexiblen Rüstung getragen werden.
    Wenn du dies tust, profitierst du von beidem.

    \subsection{Partiell}
    Die Rüstung bedeckt nicht die gesamte Trefferzone.
    Ein Gegner, der bei seinem Trefferwurf eine gerade Zahl würfelt oder einen Kritischen Treffer, ignoriert die RP der partiellen Rüstung.

    \subsection{Schwachpunkte}
    Die Rüstung hat kleine Schwachstellen, an denen eine Klinge hindurchgleiten kann, wenn der Gegner überaus geschickt ist oder einfach nur viel Glück hat.
    Wenn dein Gegner eine Waffe mit der Durchbohrend-Qualität hat und einen Kritischen Treffer erzielt, werden die RP deiner Rüstung ignoriert.

    \subsection{Undurchdringlich}
    Die Rüstung ist besonders robust, was bedeutet, dass die meisten Angriffe sie schlicht nicht durchdringen können.
    Alle Kritischen Verletzungen, die von einer ungeraden Zahl beim Trefferwurf verursacht wurden, also zum Beispiel 11 oder 33, werden ignoriert.

\end{multicols*}

\end{document}