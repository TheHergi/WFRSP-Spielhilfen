\documentclass[a4paper,10pt,twoside,twocolumn,openany,nodeprecatedcode,bg=print]{dndbook}

\usepackage[german]{babel}
\usepackage[utf8]{inputenc}
\usepackage{enumitem}

\title{
  Warhammer Fantasy Rollenspiel\newline
  \large Abänderungen und Ergänzungen zum Regelwerk
  }
\author{Patrick H.}
\date{}


\begin{document}

% \frontmatter


% \maketitle
\part*{Warhammer Fantasy Rollenspiel\newline
  \large{Abänderungen und Ergänzungen zum Regelwerk}\newline
}
\vspace{1em}

% \mainmatter%


\chapter{Allgemeine Regeln}
\section{Heldenpunkte}
\subsection{Zähigkeit und Mut}
Die Meta-Heldenpunkte Zähigkeit und Mut werden aus dem Spiel entfernt.
Damit entfällt auch die Notwendigkeit einer regeltechnischen Motivation.
Für gutes Rollenspiel sollte der Spieler aber natürliche eine Motivation finden die den Charakter antreibt.

\begin{DndComment}{Begründung}
  Diese beiden Punkte tragen unnötig zur Komplexität des Spieles bei und können weitestgehend von Glück und Schicksal abgedeckt werden.
\end{DndComment}

\subsection{Glück}
Diese beiden Meta-Heldenpunkte bleiben soweit gleich und können wie folgt vergeben werden:
% \DndItemHeader{Glück}{}
\begin{itemize}
  \item Einen Würfelwurf neu würfeln. Das Ergebnis des zweiten Wurfes ist bindend.
  \item Einen Erfolgsgrad von +1 zu einem Wurf hinzufügen.
\end{itemize}
Glückspunkte werden zu Beginn jeder Session auf die Anzahl an Schicksalspunkten aufgefüllt.

\subsection{Schicksal}
\begin{itemize}
  \item Durch eine Fügung des Schicksals überlebt dein Charakter den Kampf und stirbt nicht.
  \item Der Charakter entgeht auf eine absurde Weise jedwedem Schaden der ihn getroffen hätte.
\end{itemize}
Schicksalspunkte werden nur bei Abschluss großer Ziele, oder nach bedeutsamen Ereignissen auf Entscheidung des Meisters vergeben.

\begin{DndTable}[]{lXXXX}
  & Mensch                         &     Zwerg                    & Halbling                        & Elf \\
  Schicksalspunkte & $3$                              & $2$                            & $2$                               & $1$
\end{DndTable}

% \vspace{\fill}
% \pagebreak

% \section{Attribute}
% Der Unterschied in den Attributwerten bei der Erstellung eines Charakters ist teilweise sehr groß. 
% \begin{DndTable}[header=Attributsboni Alt]{XXXX}
%   \textbf{Volk}  & \textbf{Attributs-Offset} & \textbf{Heldenpunkte}\\
%   Mensch     & $\pm 0$ & 6  \\
%   Zwerg      & $+30$ & 4 \\
%   Halblinge  & $+20$ & 5  \\
%   Elfen  & $+80$ & 2 
% \end{DndTable}
% \begin{DndTable}[header=Attribute Alt]{XXXXXX}
%   & \textbf{Mensch} & \textbf{Zwerg} & \textbf{Halbling} & \textbf{Elf} & \textbf{Elf alt} \\
%   KG & $+20$     & $+30$    & $+10$       & $+30$  \\
%   BF & $+20$     & $+20$    & $+30$       & $+30$  \\
%   ST & $+20$     & $+20$    & $+10$       & $+20$  \\
%   WI & $+20$     & $+30$    & $+20$       & $+20$  \\
%   I  & $+20$     & $+20$    & $+20$       & $+40$  \\
%   GW & $+20$     & $+10$    & $+20$       & $+30$  \\
%   GS & $+20$     & $+30$    & $+30$       & $+40$  \\
%   IN & $+20$     & $+20$    & $+20$       & $+30$  \\
%   WK & $+20$     & $+40$    & $+30$       & $+30$  \\
%   CH & $+20$     & $+10$    & $+30$       & $+20$  \\
%   SP & $3$      & $2$     & $2$        & $1$  
% \end{DndTable}
% Als Heldenpunkte wird dabei die Summe aus Schicksalpunkten, Zähigkeitspunkten und die darüber verteilen Zusatzpunkte bezeichnet.
% Um das etwas auszugleichen werden die Attributsboni von Elfen und Zwerge gesenkt. Durch den Wegfall von Zähigkeit werden die Heldenpunkte (in diesem Fall gibt es nur mehr Schicksalspunkte) um etwa die Hälfte gesenkt. 
% \begin{DndTable}[header=Attributsboni]{lXXXX}
%      & {Mensch}                         &     {Zwerg}                    & {Halbling}                        & {Elf} \\
%   KG & $20 \rightarrow \textbf{20}$     & $30 \rightarrow \textbf{30}$   & $10 \rightarrow \textbf{10}$      & $20 \rightarrow \textbf{30}$  \\
%   BF & $20 \rightarrow \textbf{20}$     & $20 \rightarrow \textbf{20}$   & $30 \rightarrow \textbf{30}$      & $30 \rightarrow \textbf{30}$  \\
%   ST & $20 \rightarrow \textbf{20}$     & $20 \rightarrow \textbf{20}$   & $10 \rightarrow \textbf{10}$      & $20 \rightarrow \textbf{20}$  \\
%   WI & $20 \rightarrow \textbf{20}$     & $30 \rightarrow \textbf{30}$   & $20 \rightarrow \textbf{20}$      & $20 \rightarrow \textbf{20}$  \\
%   I  & $20 \rightarrow \textbf{20}$     & $20 \rightarrow \textbf{20}$   & $20 \rightarrow \textbf{20}$      & $30 \rightarrow \textbf{40}$  \\
%   GW & $20 \rightarrow \textbf{20}$     & $10 \rightarrow \textbf{10}$   & $20 \rightarrow \textbf{20}$      & $30 \rightarrow \textbf{30}$  \\
%   GS & $20 \rightarrow \textbf{20}$     & $30 \rightarrow \textbf{30}$   & $30 \rightarrow \textbf{30}$      & $20 \rightarrow \textbf{40}$  \\
%   IN & $20 \rightarrow \textbf{20}$     & $20 \rightarrow \textbf{20}$   & $20 \rightarrow \textbf{20}$      & $30 \rightarrow \textbf{30}$  \\
%   WK & $20 \rightarrow \textbf{20}$     & $40 \rightarrow \textbf{30}$   & $30 \rightarrow \textbf{30}$      & $20 \rightarrow \textbf{30}$  \\
%   CH & $20 \rightarrow \textbf{20}$     & $10 \rightarrow \textbf{10}$   & $30 \rightarrow \textbf{30}$      & $20 \rightarrow \textbf{20}$  \\
%   SP & $3$                              & $2$                            & $2$                               & $1$  
% \end{DndTable}
% Der Unterschied in den Attributen ist damit wesentlich geringer, und die Schicksalspunkte (SP) wurden mit angepasst.
% \begin{DndTable}[header=Attributsboni Neu]{lXXll}
%   \textbf{Volk}  & \textbf{Attribute Alt} & \textbf{Attribute Neu} & \textbf{SP Alt} & \textbf{SP Neu}\\
%   Mensch  & $\pm 0$ & $\pm 0$ & 6 & 3  \\
%   Zwerg  & $+30$ & $+20$ & 4 & 2  \\
%   Halbling  & $+20$ & $+20$ & 5 & 2  \\
%   Elf  & $+80$ & $+40$ & 2 & 1  \\
% \end{DndTable}
% Elfen sind den Menschen noch immer stark überlegen, Zwerge und Halblinge nur leicht. Die Schicksalspunkte dienen (wie im Regelwerk erläutert) zur Kompensierung der Attributswerte. In diesem Fall entspricht ein Attributsbonus von $+20$ einem Schicksalspunkt.

% \begin{DndComment}{Begründung}
%   Vor allem Elfen waren Super-Meschen die von Beginn an extrem mächtig waren. Mit diesen Änderungen wird der Unterschied der Rassen etwas minimiert. Der Fakt dass Menschen den anderen Rassen unterlegen sind bleibt erhalten und wird durch die höhere Anzahl an Schicksalpuntken kompensiert.
% \end{DndComment}

\section{Vorteil}
Die bestehenden Regeln aus dem \textit{GRW} werden verworfen.
Stattdessen wird die Vorteilsregelung von \textit{Andy Law} verwendet: Der SC bekommt eine Erleichterung von $+20$ auf den Test, wenn er sich dafür mehr Zeit nehmen kann. Heißt, der SC bekommt immer einen $+20$ Vorteil, außer er ist in einer Stresssituation.

\begin{DndComment}{Begründung}
  Diese Regelung vereinfacht den Vorteil auf ein simples „Kann ich mir dafür mehr Zeit nehmen? Ja oder Nein.“. Die vielen Regeln, die für Vorteil existieren, werden dadurch durch eine einzige simple Regel ersetzt.
\end{DndComment}


\vspace{\fill}
\pagebreak

\section{Karriere}
\subsection{Talente}
Talente in vorhergehenden Karrierestufen sind dem Spieler ohne Einschränkungen verfügbar und können frei gewählt werden.
Laut GRW sind diese nicht mehr verfügbar, sobald man in die Karriere verlässt.

Grundsätzlich stehen dem Spieler alle Talente aus den offiziellen Büchern offen.
Wird ein Talent außerhalb des Karriereweges gewählt, sollte das im Vorhinein mit dem Meister abgeklärt werden.

\begin{DndComment}{Begründung}
  Auch wenn ich die Spieltechnische Entscheidung verstehe, dass man mit dem Aufstieg nicht nur Vorteile bekommt, macht es logisch keinen Sinn diese Regelung beizubehalten.
\end{DndComment}


\subsection{Aufstieg}
Zum Aufstieg in die höhere Karrierestufe wird eine bestimmte Anzahl an Steigerungen in Fähigkeiten und Talenten benötigt. die innerhalb der Karriere sein müssen.
Dass sie innerhalb der Karriere sein müssen wird entfernt, und es zählen alle Steigerungen in allen Fähigkeiten.

Beispiel: Ein Handwerkslehrling besitzt die Fähigkeit \textit{Zechen} die er auch steigern muss, wenn er zum Handwerker aufsteigen will. Nicht jeder will einen Charakter spielen der viel trinkt, und vielleicht stattdessen jemanden lieber spielen der künstlerisch begabt ist. In der neuen Regelung zählen die Steigerungen in \textit{Kunst} somit auch zum Fortschritt ein Handwerker zu werden.

\begin{DndComment}{Begründung}
  Diese Restriktion ist ziemlich streng, und zwingt den Charakter auf Schienen. Damit, dass Fähigkeitssteigerungen außerhalb der Karriere noch immer das doppelte kosten, bleibt weiterhin genügend Anreiz den gewählten Pfad zu gehen, ohne das komplette System umzuwerfen.
\end{DndComment}


\clearpage
\section{Magie}

Für Magie existieren sowohl in \textit{WoM} (\textit{Winds of Magic}) als auch in \textit{AotE3} (\textit{Archives of the Empire 3}) optionale und überarbeitete Regeln.
Dieses Kapitel stellt eine Zusammenfassung über die verwendeten Regeln dar.

\subsection{Zaubern}
Der Zauberer muss einen \textbf{Sprache: Magick} Test machen, wobei er EG in Höhe des ZW des Zaubers benötigt.\\
\textbf{Kritischer Erfolg}: Würfel auf der Tabelle für Leichte Kontrollverluste. (Ausgenommen der Zauberer besitzt das Talent „Instinktive Aussprache“) Der Zauberer kann sich einen der folgenden Effekte aussuchen:
\begin{itemize}
  \item \textbf{Kritischer Zauber}: Wenn der Zauber Schaden verursacht, verursacht er zudem eine kritische Wunde.
  \item \textbf{Totale Macht}: Der Zauber kann gewirkt werden, obwohl möglicherweise nicht genug EG erzielt wurden.
  \item \textbf{Unaufhaltsame Kraft}: Der Zauber kann nicht aufgelöst („dispelled“) werden.
\end{itemize}
\textbf{Kritischer Misserfolg}: Würfel auf der Tabelle für Leichte Kontrollverluste.

\subsection{Kanalisieren}
Bei einem erfolgreichen \textbf{Kanalisieren}-Wurf erhält der Zauberer Willenskraft-Bonus EG. Diese EG werden vom ZW des Zaubers abgezogen und können den ZW maximal auf 0 reduzieren.
Jede Runde, in der der Zauberer nicht kanalisiert oder zaubert, verliert er einen EG.
Es ist nicht notwendig beim Kanalisieren den Zauber bereits „auszuwählen“.\\
\textbf{Kritischer Erfolg}: Der Zauberer erhält zusätzlich die EG des Kanalisations-Tests. Würfel auf der Tabelle für Leichte Kontrollverluste (ausgenommen der Spieler besitzt das Talent "Äthergespür").\\
\textbf{Kritischer Misserfolg}: Alle EG, die der Zauberer hält, sind verloren. Wenn die EG, die der Zauberer dadurch verloren hat, größer sind als sein Willenskraft-Bonus, würfel auf der Tabelle für Schwere Kontrollverluste, ansonsten auf der Tabelle für Leichte Kontrollverluste.

\subsection{Unterbrechung}
Wenn der Zauberer während dem Kanalisieren unterbrochen wird (z. B. er nimmt Schaden), dann muss er einen schwierigen (-20) \textbf{Besonnenheits}-Wurf ablegen. Wenn dieser nicht gelingt, verliert er alle EG und muss auf der Tabelle für Leichte Kontrollverluste würfeln.

\subsection{Magische Geschosse}
Die Trefferzone des Geschosses ergibt sich aus dem umgedrehten Würfelergebnis.\\
Der Schaden des Geschosses ergibt sich aus dem Schaden des Zaubers plus dem \textbf{Willenskraft-Bonus} des Zauberers.
Der Zauber kann auch noch, wie folgt, zusätzlich verstärkt werden.

\subsection{Zauber verstärken - Overcasting}
Wenn beim Zaubern (\textbf{Sprache: Magick}-Wurf) mehr SL als benötigt erreicht wurden, können diese wie folgt ausgegeben werden:

\begin{DndTable}[header=Zauber verstärken\\\small{AoE: Wirkungsbereich}\\\small{RW: Reichweite}]{XXXXXX}
  \textbf{EG}   & \textbf{Ziele}  & \textbf{Schaden}  & \textbf{RW} & \textbf{AoE}  & \textbf{Dauer}  \\
  $1$           & $+1$            & $+1$              & $\times 2$  & -             & -               \\
  $2$           & $+1$            & $+2$              & $\times 2$  & -             & $\times 2$      \\
  $3$           & $+1$            & $+3$              & $\times 2$  & $\times 2$    & $\times 2$      \\
  $5$           & $+2$            & $+4$              & $\times 3$  & $\times 2$    & $\times 2$      \\
  $8$           & $+2$            & $+5$              & $\times 3$  & $\times 2$    & $\times 3$      \\
  $13$          & $+2$            & $+6$              & $\times 3$  & $\times 2$    & $\times 2$      \\
  $21+$         & $+3$            & $+7$              & $\times 4$  & $\times 3$    & $\times 2$
\end{DndTable}
Jede Spalte kann dabei nur ein Mal ausgewählt werden.
\begin{DndComment}{Beispiel}
  Der Zauberer zaubert ein magisches Geschoss und erzielt dabei +4 EG.
  Er gibt +1 EG aus, um ein zusätzliches Ziel zu treffen; es verbleiben ihm +3 EG.
  Diese verbleibenden +3 EG gibt er für +3 Schaden aus.
\end{DndComment}

\subsection{Grimoire}
Wenn der Zauberer einen \textbf{kritischen Misserfolg} beim Zaubern aus dem Grimoire erleidet, würfelt er zusätzlich noch auf folgender Tabelle um zu bestimmen was mit dem Grimoire passiert.
\begin{DndTable}[header=Kontrollverlust Grimoire]{lX}
  \textbf{Wurf}   & \textbf{Effekt}  \\
  $01-02$           & \textbf{Ins Gedächtnis eingebrannt} Der Zauberer nimmt 1d10+1 Schaden (ignoriert Rüstung). Der Zauber gilt als eingeprägt und muss nicht mehr erlernt werden.           \\
  $03-10$           & \textbf{Nur Schmerz, kein Gewinn} Der Zauberer nimmt 1d10+4 Schaden (ignoriert Rüstung)         \\
  $11-90$           & \textbf{Kein Effekt}           \\
  $91-95$           & \textbf{In Rauch aufgehen} Die jeweiligen Seiten des Grimoires brennen ab.          \\
  $96-00$           & \textbf{Verbrennung} Das Grimoire ist zerstört.
\end{DndTable}


\chapter{Charaktererstellung}
Eine Schritt-für-Schritt-Anleitung zur Erstellung eines Charakters, und ausfüllen des Charakterbogens.

\subsection{1. Volk}
Wähle ein Volk zwischen Mensch, Halbling und Zwerg aus.
Andere Völker (Elfen, Gnome, Oger und Echsenmenschen) stehen vorerst nicht zur verfügung.

\begin{DndReadAloud}
  \begin{itemize}[noitemsep]
    \item Trage dein Volk unter 'Volk' ein.
  \end{itemize}
\end{DndReadAloud}

\subsection{2. Karriere}
Wähle eine Karriere für deinen Charakter.
Dein Charakter beginnt in der ersten Karrierestufe der Karriere. Wählt man z.B. als Karriere \textit{Zauberer}, so beginnt man das Spiel als \textit{Zauberlehrling}.

Eine Übersicht gibt es im GRW auf Seite 30-31, mit einer detaillierten Beschreibung ab Seite 53.
Beachte, dass nicht jede Karriere, jedem Volk zur Verfügung steht. Die verfügbaren Völker sind in der Detailbeschreibung unter dem Karrieretitel zu finden.

\begin{DndReadAloud}
  \begin{itemize}[noitemsep]
    \item Trage deine Karriere (z.B. \textit{Zauberer}) unter 'Karriere' ein.
    \item Trage deine Karrierestufe (z.B. \textit{Zauberlehrling}) unter 'Karrierestufe' sowie 'Karriereweg' ein.
    \item Trage den Status (z.B. \textit{Messing 3}) deiner aktuellen Karrierestufe unter 'Status' ein.
    \item Trage deine Klasse (z.B. \textit{Akademiker}) deiner aktuellen Karrierestufe unter 'Klasse' ein.
  \end{itemize}
\end{DndReadAloud}

\subsection{3. Spielwerte}
Bestimme deine 10 Attribute sowie deine Heldenpunkte:
\begin{DndReadAloud}
\begin{itemize}[noitemsep]
  \item Würfle 12 Mal 2W10 und notiere dir die Ergebnisse. Damit erhältst du 12 Werte zwischen 2 und 20.
  \item Streiche das höchste und niedrigste Ergebnis.
  \item Verteile die restlichen 10 Werte auf deine Spielwerte in der Reihe \textit{Anfagngswert}.
  \item Addiere die Basiswerte für dein Volk aus der Tabelle 'Basiswerte' auf den \textit{Anfagngswert} auf.
  \item Trage dir deine Schicksalspunkte aus der Tabelle 'Basiswerte' am Charakterblatt ein. Deine Glückspunkte sind gleich deiner Schicksalspunkte.
\end{itemize}
\end{DndReadAloud}

\begin{DndTable}[header=Basiswerte]{lXXXXX}
  & \textbf{Mensch} & \textbf{Zwerg} & \textbf{Halbling} \\
  KG               & $20$           & $30$          & $10$                \\
  BF               & $20$           & $20$          & $30$                \\
  ST               & $20$           & $20$          & $10$                \\
  WI               & $20$           & $30$          & $20$                \\
  I                & $20$           & $20$          & $20$                \\
  GW               & $20$           & $10$          & $20$                \\
  GS               & $20$           & $30$          & $30$                \\
  IN               & $20$           & $20$          & $20$                \\
  WK               & $20$           & $40$          & $30$                \\
  CH               & $20$           & $10$          & $30$                \\
  Schicksalspunkte & $3$            & $2$            & $2$
\end{DndTable}

\subsection[]{4. Volks-Talente}
Bestimme die fünf Talente deines Volkes. Diese sind im Grundregelwerk auf Seite 36 gelistet.

\begin{DndComment}{Zufällige Talente}
  Anstatt zufällige Talente zu würfeln, kann der Spieler diese auch selbst aus der Tabelle auswählen.
\end{DndComment}

\begin{DndComment}{Unkenruf}
  Dieses Talent ist mehr Fluff als eigentliches Talent und bietet abgesehen von einem Meta-Vorteil (Charakter stirbt) keine Vorteile beim Spielen.
  Der Spieler sucht anstatt Unkenruf ein anderes Talent aus.
\end{DndComment}

\subsection[]{5. Volks-Fähigkeiten}
Entscheide dich für 6 Fähigkeiten von deinem Volk (GRW Seite 36).
Vergebe für 3 Fähigkeiten 5 Steigerungen, und für die restlichen 3 Fähigkeiten 3 Steigerungen.

\subsection[]{6. Karriere-Talent}
Wähle ein Talent in deiner Karrierestufe.

\subsection[]{7. Karriere-Fähigkeiten}
Verteile 40 Steigerungen in die Fähigkeiten deiner Karrierestufe.
Pro Fähigkeit dürfen in diesem Schritt maximal 10 Punkte vergeben werden.


\end{document}
