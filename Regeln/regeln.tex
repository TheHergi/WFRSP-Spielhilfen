\documentclass[a4paper,10pt,twoside,twocolumn,openany,nodeprecatedcode,bg=print]{dndbook}

\usepackage[german]{babel}
\usepackage[utf8]{inputenc}
\usepackage{enumitem}
\usepackage{multirow}

\title{
  Warhammer Fantasy Rollenspiel\newline
  \large Abänderungen und Ergänzungen zum Regelwerk
  }
\author{Patrick H.}
\date{}


\begin{document}

% \frontmatter


% \maketitle
\part*{Warhammer Fantasy Rollenspiel\newline
  \large{Abänderungen und Ergänzungen zum Regelwerk}\newline
}
\vspace{1em}

% \mainmatter%


\chapter{Allgemeine Regeln}
\section{Heldenpunkte}
\subsection{Zähigkeit und Mut}
Die Meta-Heldenpunkte Zähigkeit und Mut werden aus dem Spiel entfernt.
Damit entfällt auch die Notwendigkeit einer regeltechnischen Motivation.
% Der Spieler sollte aber natürliche eine Motivation finden die den Charakter antreibt.

\begin{DndComment}{Begründung}
  Diese beiden Punkte tragen unnötig zur Komplexität des Spieles bei und können weitestgehend von Glück und Schicksal abgedeckt werden.
\end{DndComment}

\subsection{Schicksal und Glück}
Diese beiden Meta-Heldenpunkte bleiben soweit gleich und können wie folgt vergeben werden:
\subsubsection{Glück}
\begin{itemize}
  \item Einen Würfelwurf neu würfeln. Das Ergebnis des zweiten Wurfes ist bindend.
  \item Einen Erfolgsgrad von +1 zu einem Wurf hinzufügen.
\end{itemize}
Glückspunkte werden zu Beginn jeder Session auf die Anzahl an Schicksalspunkten aufgefüllt.
asd
\subsubsection{Schicksal}
\begin{itemize}
  \item Durch eine Fügung des Schicksals überlebt dein Charakter den Kampf und stirbt nicht.
  \item Dein Charakter korrumpiert nicht. Dadurch verlierst du keine Korrumpierungspunkte.
        % \item Der Charakter entgeht auf eine absurde Weise jedwedem Schaden der ihn getroffen hätte.
\end{itemize}
Schicksalspunkte werden nur bei Abschluss großer Ziele, oder nach bedeutsamen Ereignissen auf Entscheidung des Meisters vergeben.

% \begin{DndTable}[]{lXXXX}
%   & Mensch                         &     Zwerg                    & Halbling                        & Elf \\
%   Schicksalspunkte & $3$                              & $2$                            & $2$                               & $1$
% \end{DndTable}

% \vspace{\fill}
% \pagebreak

% \section{Attribute}
% Der Unterschied in den Attributwerten bei der Erstellung eines Charakters ist teilweise sehr groß. 
% \begin{DndTable}[header=Attributsboni Alt]{XXXX}
%   \textbf{Volk}  & \textbf{Attributs-Offset} & \textbf{Heldenpunkte}\\
%   Mensch     & $\pm 0$ & 6  \\
%   Zwerg      & $+30$ & 4 \\
%   Halblinge  & $+20$ & 5  \\
%   Elfen  & $+80$ & 2 
% \end{DndTable}
% \begin{DndTable}[header=Attribute Alt]{XXXXXX}
%   & \textbf{Mensch} & \textbf{Zwerg} & \textbf{Halbling} & \textbf{Elf} & \textbf{Elf alt} \\
%   KG & $+20$     & $+30$    & $+10$       & $+30$  \\
%   BF & $+20$     & $+20$    & $+30$       & $+30$  \\
%   ST & $+20$     & $+20$    & $+10$       & $+20$  \\
%   WI & $+20$     & $+30$    & $+20$       & $+20$  \\
%   I  & $+20$     & $+20$    & $+20$       & $+40$  \\
%   GW & $+20$     & $+10$    & $+20$       & $+30$  \\
%   GS & $+20$     & $+30$    & $+30$       & $+40$  \\
%   IN & $+20$     & $+20$    & $+20$       & $+30$  \\
%   WK & $+20$     & $+40$    & $+30$       & $+30$  \\
%   CH & $+20$     & $+10$    & $+30$       & $+20$  \\
%   SP & $3$      & $2$     & $2$        & $1$  
% \end{DndTable}
% Als Heldenpunkte wird dabei die Summe aus Schicksalpunkten, Zähigkeitspunkten und die darüber verteilen Zusatzpunkte bezeichnet.
% Um das etwas auszugleichen werden die Attributsboni von Elfen und Zwerge gesenkt. Durch den Wegfall von Zähigkeit werden die Heldenpunkte (in diesem Fall gibt es nur mehr Schicksalspunkte) um etwa die Hälfte gesenkt. 
% \begin{DndTable}[header=Attributsboni]{lXXXX}
%      & {Mensch}                         &     {Zwerg}                    & {Halbling}                        & {Elf} \\
%   KG & $20 \rightarrow \textbf{20}$     & $30 \rightarrow \textbf{30}$   & $10 \rightarrow \textbf{10}$      & $20 \rightarrow \textbf{30}$  \\
%   BF & $20 \rightarrow \textbf{20}$     & $20 \rightarrow \textbf{20}$   & $30 \rightarrow \textbf{30}$      & $30 \rightarrow \textbf{30}$  \\
%   ST & $20 \rightarrow \textbf{20}$     & $20 \rightarrow \textbf{20}$   & $10 \rightarrow \textbf{10}$      & $20 \rightarrow \textbf{20}$  \\
%   WI & $20 \rightarrow \textbf{20}$     & $30 \rightarrow \textbf{30}$   & $20 \rightarrow \textbf{20}$      & $20 \rightarrow \textbf{20}$  \\
%   I  & $20 \rightarrow \textbf{20}$     & $20 \rightarrow \textbf{20}$   & $20 \rightarrow \textbf{20}$      & $30 \rightarrow \textbf{40}$  \\
%   GW & $20 \rightarrow \textbf{20}$     & $10 \rightarrow \textbf{10}$   & $20 \rightarrow \textbf{20}$      & $30 \rightarrow \textbf{30}$  \\
%   GS & $20 \rightarrow \textbf{20}$     & $30 \rightarrow \textbf{30}$   & $30 \rightarrow \textbf{30}$      & $20 \rightarrow \textbf{40}$  \\
%   IN & $20 \rightarrow \textbf{20}$     & $20 \rightarrow \textbf{20}$   & $20 \rightarrow \textbf{20}$      & $30 \rightarrow \textbf{30}$  \\
%   WK & $20 \rightarrow \textbf{20}$     & $40 \rightarrow \textbf{30}$   & $30 \rightarrow \textbf{30}$      & $20 \rightarrow \textbf{30}$  \\
%   CH & $20 \rightarrow \textbf{20}$     & $10 \rightarrow \textbf{10}$   & $30 \rightarrow \textbf{30}$      & $20 \rightarrow \textbf{20}$  \\
%   SP & $3$                              & $2$                            & $2$                               & $1$  
% \end{DndTable}
% Der Unterschied in den Attributen ist damit wesentlich geringer, und die Schicksalspunkte (SP) wurden mit angepasst.
% \begin{DndTable}[header=Attributsboni Neu]{lXXll}
%   \textbf{Volk}  & \textbf{Attribute Alt} & \textbf{Attribute Neu} & \textbf{SP Alt} & \textbf{SP Neu}\\
%   Mensch  & $\pm 0$ & $\pm 0$ & 6 & 3  \\
%   Zwerg  & $+30$ & $+20$ & 4 & 2  \\
%   Halbling  & $+20$ & $+20$ & 5 & 2  \\
%   Elf  & $+80$ & $+40$ & 2 & 1  \\
% \end{DndTable}
% Elfen sind den Menschen noch immer stark überlegen, Zwerge und Halblinge nur leicht. Die Schicksalspunkte dienen (wie im Regelwerk erläutert) zur Kompensierung der Attributswerte. In diesem Fall entspricht ein Attributsbonus von $+20$ einem Schicksalspunkt.

% \begin{DndComment}{Begründung}
%   Vor allem Elfen waren Super-Meschen die von Beginn an extrem mächtig waren. Mit diesen Änderungen wird der Unterschied der Rassen etwas minimiert. Der Fakt dass Menschen den anderen Rassen unterlegen sind bleibt erhalten und wird durch die höhere Anzahl an Schicksalpuntken kompensiert.
% \end{DndComment}

\section{Vorteil}
Die bestehenden Regeln aus dem \textit{GRW} werden verworfen.
Stattdessen wird die Vorteilsregelung von \textit{Lawhammer} verwendet: Der SC bekommt eine Erleichterung von $+20$ auf den Test, wenn er sich dafür mehr Zeit nehmen kann. Heißt, der SC bekommt immer einen $+20$ Vorteil, außer er ist in einer Stresssituation.

\begin{DndComment}{Begründung}
  Diese Regelung vereinfacht den Vorteil auf ein simples „Kann ich mir dafür mehr Zeit nehmen? Ja oder Nein.“. Die vielen Regeln, die für Vorteil existieren, werden dadurch durch eine einzige simple Regel ersetzt.
\end{DndComment}


\vspace{\fill}
\pagebreak

\section{Karriere}
Mit dem Erwerb von Erfahrungspunkten (EP) können
\begin{itemize}
  \item Steigerungen in Attribute (Spielwerte)
  \item Steigerungen in Fähigkeiten
  \item Talente
\end{itemize}
gekauft werden.

\subsection[short]{Attribute und Fähigkeiten}

Die Kosten zum Steigern eines Attributes (Spielwert) oder einer Fähigkeit sind in der folgenden Tabelle ersichtlich. Die Tabelle listet dabei die Kosten um ein Attribut oder eine Fähigkeit um 1 zu steigern.

Der angegebene Preis gilt für Attribute und Fähigkeiten die dem Charakter durch die Karriere oder dem Volk zugeordnet worden sind. (Auf dem Charakterblatt wurden diese extra markiert)
Möchte man eine Steigerung außerhalb der Karriere oder des Volkes vornehmen, so verdoppeln sich die angegebenen Kosten.
\begin{DndTable}[header=EP-Kosten]{XXX}
  \textbf{Steigerungen}                              & \textbf{Attribut}             & \textbf{Fähigkeit}             \\
  0-5                           & 25                   & 10                    \\
  6-10                          & 30                   & 15                    \\
  11-15                         & 40                   & 20                    \\
  16-20                         & 50                   & 30                    \\
  21-25                         & 70                   & 40                    \\
  26-30                         & 90                   & 60                    \\
  31-35                         & 120                  & 80                    \\
  36-40                         & 150                  & 110                   \\
  41-45                         & 190                  & 140                   \\
  46-50                         & 230                  & 180
\end{DndTable}


\textbf{Beispiel 1}: Du hast bereits 8 Steigerungen im Attribut \textit{Kampfgeschick}. Die Steigerung von 8 auf 9 Steigerungen kostet dich 30 EP.\newline
\textbf{Beispiel 2}: Du hast 15 Steigerungen in der Fähigkeit \textit{Ausweichen}. Die Steigerung von 15 auf 16 kostet dich 20 EP.\\
\textbf{Beispiel 3}: Du spielst einen Laienpriester (Kriegerpriester) und möchtest die Fähigkeit \textit{Bestechen} steigern in der du 0 Steigerungen hast. Diese Fähigkeit ist weder Teil deiner Karriere noch deines Volkes, also musst du 20, statt 10 EP zahlen um die Fähigkeit von 0 auf 1 zu steigern.

\begin{DndComment}{}
  Laut GRW kosten auch die Steigerungen der Fähigkeiten deines Volkes doppelt. Diese Restriktion wurde für mehr Freiheit fallen gelassen.
\end{DndComment}

\subsection{Talente}
Ein neues Talent zu erlernen kostet 100 EP. Bei einigen Talenten ist es möglich dieses mehrere Male zu kaufen/steigern. In diesem Fall kostet die Steigerung des Talentes 100 EP + 100 EP für jedes Mal, dass dieses Talent bereits gelernt wurde.\\

Talente in der aktuellen, und der vorhergehenden Karrierestufen sind dem Spieler ohne Einschränkungen verfügbar und können frei gewählt werden.
Grundsätzlich stehen dem Spieler alle Talente aus den Büchern offen, sofern sie Sinnvoll für den Charakter sind.
Talente können eine äußerst starke Auswirkungen auf das Spiel haben, oder können nur über längere Ingame-Zeit erlernt werden (z.B. das Erlernen einer Sprache). 
Deswegen wird vor Erwerb eines Talentes um eine kurze Rücksprache mit dem Spielleiter gebeten.

\begin{DndComment}{Begründung}
  Laut GRW sind nur Talente in der aktuellen Karrierestufen verfügbar. Diese Einschränkung ist sehr restriktiv und wird fallen gelassen.
\end{DndComment}


\subsection{Karriere-Aufstieg}
Um in der Karriereleiter aufzusteigen, muss zuerst die vorherige Karrierestufe abgeschlossen werden. Dafür wird eine Mindestanzahl an Steigerungen in Fähigkeiten und Attributen, sowie eine Mindestanzahl an Talenten benötigt.

\begin{DndTable}[header=Abschließen einer Karrierestufe]{llXXX}
  \multicolumn{2}{c}{Karrierestufe} & \multicolumn{2}{c}{Steigerungen} &         \\
  Stufe      & Beispiel             & Attribute      & Fähigkeiten     & Talente \\
  1     & Laienpriester                                                       & 15        & 40          & 1       \\
  2     & Kriegerpriester                                                     & 40        & 80          & 2       \\
  3     & Feldpriester                                                        & 75        & 120         & 3       \\
  4     & Priestergeneral                                                     & 120       & 160         & 4      
\end{DndTable}

Die Steigerungen können dabei in beliebigen Attributen oder Fähigkeiten sein; Steigerungen und Talente aus der Charaktererstellung zählen damit auch mit hinein.
Laut GRW gibt es auch EP-Kosten für das Abschließen und Starten einer neuen Karrierestufe. Auf diese zusätzlichen Kosten wird verzichtet.\\

\textbf{Beispiel}: Um die Karrierestufe Laienpriester abzuschließen und die Karrierestufe Kriegerpriester zu starten werden insgesamt 15 Steigerungen in Attributen, 40 Steigerungen in Fähigkeiten, und mindestens 1 Talent benötigt.

\begin{DndComment}{Begründung}
  Laut GRW werden zählen nur Steigerungen deiner aktuellen Karrierestufe, diese Restriktion ist ziemlich streng, und zwingt den Charakter sehr auf Schienen. Damit, dass Steigerungen außerhalb der Karriere noch immer das doppelte kosten, bleibt weiterhin genügend Anreiz den gewählten Pfad zu gehen, ohne das komplette System umzuwerfen.
\end{DndComment}


\clearpage
% \section{Magie}

% Für Magie existieren sowohl in \textit{WoM} (\textit{Winds of Magic}) als auch in \textit{AotE3} (\textit{Archives of the Empire 3}) optionale und überarbeitete Regeln.
% Dieses Kapitel stellt eine Zusammenfassung über die verwendeten Regeln dar.

% \subsection{Zaubern}
% Der Zauberer muss einen \textbf{Sprache: Magick} Test machen, wobei er EG in Höhe des ZW des Zaubers benötigt.\\
% \textbf{Kritischer Erfolg}: Würfel auf der Tabelle für Leichte Kontrollverluste. (Ausgenommen der Zauberer besitzt das Talent „Instinktive Aussprache“) Der Zauberer kann sich einen der folgenden Effekte aussuchen:
% \begin{itemize}
%   \item \textbf{Kritischer Zauber}: Wenn der Zauber Schaden verursacht, verursacht er zudem eine kritische Wunde.
%   \item \textbf{Totale Macht}: Der Zauber kann gewirkt werden, obwohl möglicherweise nicht genug EG erzielt wurden.
%   \item \textbf{Unaufhaltsame Kraft}: Der Zauber kann nicht aufgelöst („dispelled“) werden.
% \end{itemize}
% \textbf{Kritischer Misserfolg}: Würfel auf der Tabelle für Leichte Kontrollverluste.

% \subsection{Kanalisieren}
% Bei einem erfolgreichen \textbf{Kanalisieren}-Wurf erhält der Zauberer Willenskraft-Bonus EG. Diese EG werden vom ZW des Zaubers abgezogen und können den ZW maximal auf 0 reduzieren.
% Jede Runde, in der der Zauberer nicht kanalisiert oder zaubert, verliert er einen EG.
% Es ist nicht notwendig beim Kanalisieren den Zauber bereits „auszuwählen“.\\
% \textbf{Kritischer Erfolg}: Der Zauberer erhält zusätzlich die EG des Kanalisations-Tests. Würfel auf der Tabelle für Leichte Kontrollverluste (ausgenommen der Spieler besitzt das Talent "Äthergespür").\\
% \textbf{Kritischer Misserfolg}: Alle EG, die der Zauberer hält, sind verloren. Wenn die EG, die der Zauberer dadurch verloren hat, größer sind als sein Willenskraft-Bonus, würfel auf der Tabelle für Schwere Kontrollverluste, ansonsten auf der Tabelle für Leichte Kontrollverluste.

% \subsection{Unterbrechung}
% Wenn der Zauberer während dem Kanalisieren unterbrochen wird (z. B. er nimmt Schaden), dann muss er einen schwierigen (-20) \textbf{Besonnenheits}-Wurf ablegen. Wenn dieser nicht gelingt, verliert er alle EG und muss auf der Tabelle für Leichte Kontrollverluste würfeln.

% \subsection{Magische Geschosse}
% Die Trefferzone des Geschosses ergibt sich aus dem umgedrehten Würfelergebnis.\\
% Der Schaden des Geschosses ergibt sich aus dem Schaden des Zaubers plus dem \textbf{Willenskraft-Bonus} des Zauberers.
% Der Zauber kann auch noch, wie folgt, zusätzlich verstärkt werden.

% \subsection{Zauber verstärken - Overcasting}
% Wenn beim Zaubern (\textbf{Sprache: Magick}-Wurf) mehr SL als benötigt erreicht wurden, können diese wie folgt ausgegeben werden:

% \begin{DndTable}[header=Zauber verstärken\\\small{AoE: Wirkungsbereich}\\\small{RW: Reichweite}]{lccccX}
%   \textbf{EG}   & \textbf{Ziele}  & \textbf{Schaden}  & \textbf{RW} & \textbf{AoE}  & \textbf{Dauer}  \\
%   $1$           & $+1$            & $+1$              & $\times 2$  & -             & -               \\
%   $2$           & $+1$            & $+2$              & $\times 2$  & -             & $\times 2$      \\
%   $3$           & $+1$            & $+3$              & $\times 2$  & $\times 2$    & $\times 2$      \\
%   $5$           & $+2$            & $+4$              & $\times 3$  & $\times 2$    & $\times 2$      \\
%   $8$           & $+2$            & $+5$              & $\times 3$  & $\times 2$    & $\times 3$      \\
%   $13$          & $+2$            & $+6$              & $\times 3$  & $\times 2$    & $\times 2$      \\
%   $21+$         & $+3$            & $+7$              & $\times 4$  & $\times 3$    & $\times 2$
% \end{DndTable}
% Jede Spalte kann dabei nur ein Mal ausgewählt werden.
% \begin{DndComment}{Beispiel}
%   Der Zauberer zaubert ein magisches Geschoss und erzielt dabei +4 EG.
%   Er gibt +1 EG aus, um ein zusätzliches Ziel zu treffen; es verbleiben ihm +3 EG.
%   Diese verbleibenden +3 EG gibt er für +3 Schaden aus.
% \end{DndComment}

% \subsection{Grimoire}
% Wenn der Zauberer einen \textbf{kritischen Misserfolg} beim Zaubern aus dem Grimoire erleidet, würfelt er zusätzlich noch auf folgender Tabelle um zu bestimmen was mit dem Grimoire passiert.
% \begin{DndTable}[header=Kontrollverlust Grimoire]{lX}
%   \textbf{Wurf}   & \textbf{Effekt}  \\
%   $01-02$           & \textbf{Ins Gedächtnis eingebrannt} Der Zauberer nimmt 1d10+1 Schaden (ignoriert Rüstung). Der Zauber gilt als eingeprägt und muss nicht mehr erlernt werden.           \\
%   $03-10$           & \textbf{Nur Schmerz, kein Gewinn} Der Zauberer nimmt 1d10+4 Schaden (ignoriert Rüstung)         \\
%   $11-90$           & \textbf{Kein Effekt}           \\
%   $91-95$           & \textbf{In Rauch aufgehen} Die jeweiligen Seiten des Grimoires brennen ab.          \\
%   $96-00$           & \textbf{Verbrennung} Das Grimoire ist zerstört.
% \end{DndTable}


\chapter{Charaktererstellung}
% Eine Schritt-für-Schritt-Anleitung zur Erstellung eines Charakters.

\subsection{1. Volk}
Wähle ein Volk zwischen Mensch, Halbling und Zwerg aus.
Andere Völker (Elfen, Gnome, Oger und Echsenmenschen) stehen vorerst nicht zur Verfügung.


\subsection{2. Karriere}
Wähle eine Karriere für deinen Charakter und fülle die Kategorie \textit{Karriere} am Charakterblatt aus.
Dein Charakter beginnt in der ersten Karrierestufe der Karriere.
Trage den Rang und das Ansehen deiner Karrierestufe ein.


% Eine Übersicht gibt es im GRW auf Seite 30-31, mit einer detaillierten Beschreibung ab Seite 53.
% Beachte, dass nicht jede Karriere, jedem Volk zur Verfügung steht. Die verfügbaren Völker sind in der Detailbeschreibung unter dem Karrieretitel zu finden.

% \begin{DndReadAloud}
%   \begin{itemize}[noitemsep]
%     \item Trage deine Karriere (z.B. \textit{Zauberer}) unter 'Karriere' ein.
%     \item Trage deine Karrierestufe (z.B. \textit{Zauberlehrling}) unter 'Karrierestufe' sowie 'Karriereweg' ein.
%     \item Trage den Status (z.B. \textit{Messing 3}) deiner aktuellen Karrierestufe unter 'Status' ein.
%     \item Trage deine Klasse (z.B. \textit{Akademiker}) deiner aktuellen Karrierestufe unter 'Klasse' ein.
%   \end{itemize}
% \end{DndReadAloud}

\subsection{3. Spielwerte}
Bestimme deine 10 Attribute:
% \begin{DndReadAloud}
\begin{itemize}[noitemsep]
  \item Würfle 12 Mal 2W10 und notiere dir die Ergebnisse. Damit erhältst du 12 Werte zwischen 2 und 20.
  \item Streiche das höchste und niedrigste Ergebnis.
  \item Verteile die restlichen 10 Werte auf deine Spielwerte in der Reihe \textit{Anfagngswert}.
  \item Addiere die Basiswerte für dein Volk aus der Tabelle \textit{Basiswerte} auf den \textit{Anfagngswert} auf.
  \item Verteile insgesamt 5 Steigerungen in der Reihe \textit{Steigerungen} aus den Attributen deiner Karrierestufe.
  \item Trage dir deine Schicksalspunkte aus der Tabelle \textit{Basiswerte} am Charakterblatt ein. Deine Glückspunkte sind gleich deiner Schicksalspunkte.
\end{itemize}
% \end{DndReadAloud}

\begin{DndTable}[header=Basiswerte]{lXXXXX}
  & \textbf{Mensch} & \textbf{Zwerg} & \textbf{Halbling} \\
  KG               & $20$           & $30$          & $10$                \\
  BF               & $20$           & $20$          & $20$                \\
  ST               & $20$           & $20$          & $10$                \\
  WI               & $20$           & $30$          & $20$                \\
  I                & $20$           & $20$          & $30$                \\
  GW               & $20$           & $10$          & $20$                \\
  GS               & $20$           & $30$          & $30$                \\
  IN               & $20$           & $20$          & $20$                \\
  WK               & $20$           & $30$          & $30$                \\
  CH               & $20$           & $10$          & $30$                \\
  Schicksalspunkte & $3$            & $2$            & $2$
\end{DndTable}
Anmerkung: \textit{BF} und \textit{I} wurden für Halblinge vertauscht.
Für Zwerge wurde der \textit{WK}-Bonus von 40 auf 30 reduziert.

\subsection[]{4. Volks-Talente}
Bestimme die fünf Talente deines Volkes. Diese sind im Grundregelwerk auf Seite 36 gelistet.
Anstatt zufällige Talente zu würfeln, kann der Spieler diese auch selbst aus der Tabelle auswählen.\\
Die Tabelle wird mit den Talenten \textit{Gesellig}, \textit{Beharrlich} und \textit{Genügsam} erweitert.

\begin{DndComment}{Unkenruf}
  Dieses Talent ist mehr Fluff als eigentliches Talent und bietet abgesehen von einem Meta-Vorteil (Charakter stirbt) keine Vorteile beim Spielen.
  Der Spieler sucht anstatt Unkenruf ein anderes Talent aus.
\end{DndComment}

\subsection[]{5. Volks-Fähigkeiten}
Entscheide dich für 6 Fähigkeiten von deinem Volk (GRW Seite 36).
Vergebe für 3 Fähigkeiten 5 Steigerungen, und für die restlichen 3 Fähigkeiten 3 Steigerungen.
Notiere dir \textit{Sprache (Reikspiel)} mit $0$ Steigerungen.

\subsection[]{6. Karriere-Talent}
Wähle ein Talent in deiner Karrierestufe.

\subsection[]{7. Karriere-Fähigkeiten}
Verteile 40 Steigerungen in die Fähigkeiten deiner Karrierestufe.
Pro Fähigkeit dürfen in diesem Schritt maximal 10 Punkte vergeben werden.

\subsection[]{8. Ausrüstung}
Du bekommst die Ausrüstung deiner Klasse (Seite 37) und deiner ersten Karrierestufe.
Die angegebenen Gegestände sind ein Beispiel. Der Spieler ist angehalten seine Ausrüstung (sinnvoll) zu erweitern.\\
Zusätzlich erhält der Spieler ein Startgeld:
\begin{DndTable}[header=Startgeld]{lX}
  \textbf{Rang} & \textbf{Vermögen} \\
  Messing               & 2W10 Messinggroschen pro Ansehens-Stufe                 \\
  Silber               & 1W10 Silberschilling pro Ansehens-Stufe                 \\
  Gold & 1 Goldkrone pro Ansehens-Stufe
\end{DndTable}

\subsection[]{9. Werte}
\begin{itemize}
  \item Fülle die Reihe \textit{Wert} deiner \textit{Attribute} und \textit{Fähigkeiten} aus.
  \item Der Spieler beginnt mit 0 Korrumpierungspunkten.
  \item Berechne deine Lebenspunkte (siehe Seite 33).
\end{itemize}


\subsection[]{10. Details}
Fülle deine Charakterdetails aus:
\begin{itemize}
  \item Alter
  \item Haarfarbe
  \item Augenfarbe
  \item Körpergröße
  \item Bewegung
\end{itemize}

Makiere dir deine \textit{Attribute} und \textit{Fähigkeiten} fürs Steigern.

\end{document}
