\documentclass[a4paper,10pt,twoside,twocolumn,openany,nodeprecatedcode,bg=print]{dndbook}

\usepackage[german]{babel}
\usepackage[utf8]{inputenc}
\usepackage{enumitem}
\usepackage{multirow}
\usepackage[singlelinecheck=false]{caption}
\usepackage{lipsum}
\usepackage{listings}
\usepackage{shortvrb}
\usepackage{stfloats}

\MakeShortVerb{|}

\lstset{%
  basicstyle=\ttfamily,
  language=[LaTeX]{TeX},
  breaklines=true,
}

\title{
  Warhammer Fantasy Rollenspiel\newline
  \large Ergänzungen und Erweiterungen zum Regelwerk
  }
\author{Patrick H.}
\date{}


\definecolor {WhiteColor}  {HTML} {FFFFFF} % DMG Part 3
\definecolor {bgtan}    {HTML} {FFFFFF}	% DndReadAloud background
\definecolor {bgtan2018}{HTML} {FFFFFF} % lighter version of bgtan from the 2018 Basic Rules
\definecolor {PhbLightGreen} {HTML} {FFFFFF} % PHB Part 1

\begin{document}


\frontmatter

\maketitle
% \part*{Warhammer Fantasy Rollenspiel\newline
%   \large{Zusammenfassung und Ergänzung zum Regelwerk}}
% \vspace{1em}



% \mainmatter



\chapter{Charaktere}
\section{Heldenpunkte}
Als Heldenpunkte werden die Meta-Spielwerte Zähigkeit, Mut, Schicksal und Glück bezeichnet.

% \subsection{Zähigkeit und Mut}
\begin{DndReadAloud}[color=WhiteColor]
  Die Meta-Heldenpunkte Zähigkeit und Mut werden aus dem Spiel entfernt.
  \begin{DndComment}{Begründung}
    Diese beiden Punkte tragen unnötig zur Komplexität des Spieles bei und können weitestgehend von Glück und Schicksal abgedeckt werden.
  \end{DndComment}
\end{DndReadAloud}

\noindent
Damit entfällt auch die Notwendigkeit einer regeltechnischen Motivation.
Der Spieler sollte aber natürliche eine Motivation finden die den Charakter antreibt.



\subsection{Schicksal und Glück}
Diese beiden Meta-Heldenpunkte bleiben soweit gleich und können wie folgt vergeben werden:
\subsubsection{Glück}
\begin{itemize}
  \item Einen Würfelwurf neu würfeln. Das Ergebnis des zweiten Wurfes ist bindend.
  \item Einen Erfolgsgrad von +1 zu einem Wurf hinzufügen.
\end{itemize}
Glückspunkte werden zu Beginn jeder Session auf die Anzahl an Schicksalspunkten aufgefüllt.

\subsubsection{Schicksal}
\begin{itemize}
  \item Durch eine Fügung des Schicksals überlebt dein Charakter den Kampf und stirbt nicht.
  \item Dein Charakter korrumpiert nicht. Dadurch verlierst du keine Korrumpierungspunkte.
        % \item Der Charakter entgeht auf eine absurde Weise jedwedem Schaden der ihn getroffen hätte.
\end{itemize}
Schicksalspunkte werden nur bei Abschluss großer Ziele, oder nach bedeutsamen Ereignissen vergeben.

\section{Vorteil}
Die bestehenden Regeln aus dem \textit{GRW} werden verworfen.
Stattdessen wird die Vorteilsregelung von \textit{Lawhammer} verwendet: Der SC bekommt eine Erleichterung von $+20$ auf den Test, wenn er sich dafür mehr Zeit nehmen kann. Heißt, der SC bekommt immer einen $+20$ Vorteil, außer er ist in einer Stresssituation.

\begin{DndComment}{Begründung}
  Diese Regelung vereinfacht den Vorteil auf ein simples „Kann ich mir dafür mehr Zeit nehmen? Ja oder Nein“. Die vielen Regeln, die für Vorteil existieren, werden dadurch durch eine einzige simple Regel ersetzt.
\end{DndComment}


\vspace{\fill}
\pagebreak

\section{Karriere}
Mit dem Erwerb von Erfahrungspunkten (EP) können
\begin{itemize}
  \item Steigerungen in Spielwerte (Attribute)
  \item Steigerungen in Fähigkeiten
  \item Talente
\end{itemize}
gekauft werden.

\subsection[short]{Spielwerte und Fähigkeiten}

Die Kosten zum Steigern eines Spielwertes oder einer Fähigkeit sind in der folgenden Tabelle ersichtlich. Die Tabelle listet dabei die Kosten um ein Attribut oder eine Fähigkeit um 1 zu steigern.

\begin{DndTable}[header=\centerline{EP-Kosten beim Steigern}]{XXX}
  \textbf{Aktuelle Anzahl an Steigerungen}                              & \textbf{EP-Kosten für Spielwerte}             & \textbf{EP-Kosten für Fähigkeiten}             \\
  0-5                           & 25                   & 10                    \\
  6-10                          & 30                   & 15                    \\
  11-15                         & 40                   & 20                    \\
  16-20                         & 50                   & 30                    \\
  21-25                         & 70                   & 40                    \\
  26-30                         & 90                   & 60                    \\
  31-35                         & 120                  & 80                    \\
  36-40                         & 150                  & 110                   \\
  41-45                         & 190                  & 140                   \\
  46-50                         & 230                  & 180
\end{DndTable}

\noindent
Steigerungen von Spielwerten und Fähigkeiten außerhalb der Karriere kosten
die doppelte Zahl an angeführten EP-Kosten.


\subsection{Talente}

Das Erlernen eines neuen Talentes aus seiner aktuellen, oder vorherigen Karrierestufe kostet 100 EP.
Bei einigen Talenten ist es möglich dieses mehrere Male zu kaufen/steigern. In diesem Fall kostet die Steigerung des Talentes 100 EP + 100 EP für jedes Mal, dass dieses Talent bereits gelernt wurde.

\begin{DndComment}{Begründung}
  Laut GRW sind nur Talente in der aktuellen Karrierestufen verfügbar. Diese Einschränkung ist sehr restriktiv und wird fallen gelassen.
\end{DndComment}

\noindent
Grundsätzlich kann jeder Spieler jedes Talent (auch jene außerhalb der Karrierestufe) erlernen, sofern es sinnvoll für seinen Charakter ist. Zu beachten ist, dass wie bei den Fähigkeiten und Spielwerten, sich die EP-Kosten für das Talent verdoppeln!

\vspace{1ex}
\noindent
Talente können eine äußerst starke Auswirkungen auf das Spiel haben, oder können nur über längere Ingame-Zeit erlernt werden (z.B. das Erlernen einer Sprache). 
Deswegen wird vor Erwerb eines Talentes um eine kurze Rücksprache mit der Spielleitung gebeten.

\subsection{Karriere-Aufstieg}
Um in der Karriereleiter aufzusteigen, musst du in jedem der in deiner Karrierestufe steigerbaren Spielwert und in acht der auf deiner Karrierestufe zur Verfügung stehenden Fähigkeiten eine bestimmte Anzahl von Steigerungen haben.

Zusätzlich musst du mindestens ein Talent deiner gegenwärtigen Karrierestufe gelernt haben. Bei alldem zählen auch Fähigkeiten und Talente, die du vor dem Eintritt in die gegenwärtige Karriere erworben hast, sofern es dieselben sind.

\begin{DndTable}[header=Abschließen einer Karrierestufe]{lXX}
  \multicolumn{2}{c}{Karrierestufe} & Anzahl an Steigerungen  \\
  Stufe     & Beispiel              & \\
  1         & Laienpriester         & 5             \\
  2         & Kriegerpriester       & 10            \\
  3         & Feldpriester          & 15            \\
  4         & Priestergeneral       & 20            
\end{DndTable}

\vspace{\fill}
\pagebreak

\section{Regeln}
\subsection{}
\subsection{Kampf}
\subsubsection{Trefferzone}
\subsubsection{Kritische Treffer}

\clearpage
% \section{Magie}

% Für Magie existieren sowohl in \textit{WoM} (\textit{Winds of Magic}) als auch in \textit{AotE3} (\textit{Archives of the Empire 3}) optionale und überarbeitete Regeln.
% Dieses Kapitel stellt eine Zusammenfassung über die verwendeten Regeln dar.

% \subsection{Zaubern}
% Der Zauberer muss einen \textbf{Sprache: Magick} Test machen, wobei er EG in Höhe des ZW des Zaubers benötigt.\\
% \textbf{Kritischer Erfolg}: Würfel auf der Tabelle für Leichte Kontrollverluste. (Ausgenommen der Zauberer besitzt das Talent „Instinktive Aussprache“) Der Zauberer kann sich einen der folgenden Effekte aussuchen:
% \begin{itemize}
%   \item \textbf{Kritischer Zauber}: Wenn der Zauber Schaden verursacht, verursacht er zudem eine kritische Wunde.
%   \item \textbf{Totale Macht}: Der Zauber kann gewirkt werden, obwohl möglicherweise nicht genug EG erzielt wurden.
%   \item \textbf{Unaufhaltsame Kraft}: Der Zauber kann nicht aufgelöst („dispelled“) werden.
% \end{itemize}
% \textbf{Kritischer Misserfolg}: Würfel auf der Tabelle für Leichte Kontrollverluste.

% \subsection{Kanalisieren}
% Bei einem erfolgreichen \textbf{Kanalisieren}-Wurf erhält der Zauberer Willenskraft-Bonus EG. Diese EG werden vom ZW des Zaubers abgezogen und können den ZW maximal auf 0 reduzieren.
% Jede Runde, in der der Zauberer nicht kanalisiert oder zaubert, verliert er einen EG.
% Es ist nicht notwendig beim Kanalisieren den Zauber bereits „auszuwählen“.\\
% \textbf{Kritischer Erfolg}: Der Zauberer erhält zusätzlich die EG des Kanalisations-Tests. Würfel auf der Tabelle für Leichte Kontrollverluste (ausgenommen der Spieler besitzt das Talent "Äthergespür").\\
% \textbf{Kritischer Misserfolg}: Alle EG, die der Zauberer hält, sind verloren. Wenn die EG, die der Zauberer dadurch verloren hat, größer sind als sein Willenskraft-Bonus, würfel auf der Tabelle für Schwere Kontrollverluste, ansonsten auf der Tabelle für Leichte Kontrollverluste.

% \subsection{Unterbrechung}
% Wenn der Zauberer während dem Kanalisieren unterbrochen wird (z. B. er nimmt Schaden), dann muss er einen schwierigen (-20) \textbf{Besonnenheits}-Wurf ablegen. Wenn dieser nicht gelingt, verliert er alle EG und muss auf der Tabelle für Leichte Kontrollverluste würfeln.

% \subsection{Magische Geschosse}
% Die Trefferzone des Geschosses ergibt sich aus dem umgedrehten Würfelergebnis.\\
% Der Schaden des Geschosses ergibt sich aus dem Schaden des Zaubers plus dem \textbf{Willenskraft-Bonus} des Zauberers.
% Der Zauber kann auch noch, wie folgt, zusätzlich verstärkt werden.

% \subsection{Zauber verstärken - Overcasting}
% Wenn beim Zaubern (\textbf{Sprache: Magick}-Wurf) mehr SL als benötigt erreicht wurden, können diese wie folgt ausgegeben werden:

% \begin{DndTable}[header=Zauber verstärken\\\small{AoE: Wirkungsbereich}\\\small{RW: Reichweite}]{lccccX}
%   \textbf{EG}   & \textbf{Ziele}  & \textbf{Schaden}  & \textbf{RW} & \textbf{AoE}  & \textbf{Dauer}  \\
%   $1$           & $+1$            & $+1$              & $\times 2$  & -             & -               \\
%   $2$           & $+1$            & $+2$              & $\times 2$  & -             & $\times 2$      \\
%   $3$           & $+1$            & $+3$              & $\times 2$  & $\times 2$    & $\times 2$      \\
%   $5$           & $+2$            & $+4$              & $\times 3$  & $\times 2$    & $\times 2$      \\
%   $8$           & $+2$            & $+5$              & $\times 3$  & $\times 2$    & $\times 3$      \\
%   $13$          & $+2$            & $+6$              & $\times 3$  & $\times 2$    & $\times 2$      \\
%   $21+$         & $+3$            & $+7$              & $\times 4$  & $\times 3$    & $\times 2$
% \end{DndTable}
% Jede Spalte kann dabei nur ein Mal ausgewählt werden.
% \begin{DndComment}{Beispiel}
%   Der Zauberer zaubert ein magisches Geschoss und erzielt dabei +4 EG.
%   Er gibt +1 EG aus, um ein zusätzliches Ziel zu treffen; es verbleiben ihm +3 EG.
%   Diese verbleibenden +3 EG gibt er für +3 Schaden aus.
% \end{DndComment}

% \subsection{Grimoire}
% Wenn der Zauberer einen \textbf{kritischen Misserfolg} beim Zaubern aus dem Grimoire erleidet, würfelt er zusätzlich noch auf folgender Tabelle um zu bestimmen was mit dem Grimoire passiert.
% \begin{DndTable}[header=Kontrollverlust Grimoire]{lX}
%   \textbf{Wurf}   & \textbf{Effekt}  \\
%   $01-02$           & \textbf{Ins Gedächtnis eingebrannt} Der Zauberer nimmt 1d10+1 Schaden (ignoriert Rüstung). Der Zauber gilt als eingeprägt und muss nicht mehr erlernt werden.           \\
%   $03-10$           & \textbf{Nur Schmerz, kein Gewinn} Der Zauberer nimmt 1d10+4 Schaden (ignoriert Rüstung)         \\
%   $11-90$           & \textbf{Kein Effekt}           \\
%   $91-95$           & \textbf{In Rauch aufgehen} Die jeweiligen Seiten des Grimoires brennen ab.          \\
%   $96-00$           & \textbf{Verbrennung} Das Grimoire ist zerstört.
% \end{DndTable}


\chapter{Charaktererstellung}
% Eine Schritt-für-Schritt-Anleitung zur Erstellung eines Charakters.

\subsection{1. Volk}
Wähle ein Volk zwischen Mensch, Halbling und Zwerg aus.
Andere Völker (Elfen, Gnome, Oger und Echsenmenschen) stehen vorerst nicht zur Verfügung.


\subsection{2. Karriere}
Wähle eine Karriere für deinen Charakter und fülle die Kategorie \textit{Karriere} am Charakterblatt aus.
Dein Charakter beginnt in der ersten Karrierestufe der Karriere.
Trage den Rang und das Ansehen deiner Karrierestufe ein.


% Eine Übersicht gibt es im GRW auf Seite 30-31, mit einer detaillierten Beschreibung ab Seite 53.
% Beachte, dass nicht jede Karriere, jedem Volk zur Verfügung steht. Die verfügbaren Völker sind in der Detailbeschreibung unter dem Karrieretitel zu finden.

% \begin{DndReadAloud}
%   \begin{itemize}[noitemsep]
%     \item Trage deine Karriere (z.B. \textit{Zauberer}) unter 'Karriere' ein.
%     \item Trage deine Karrierestufe (z.B. \textit{Zauberlehrling}) unter 'Karrierestufe' sowie 'Karriereweg' ein.
%     \item Trage den Status (z.B. \textit{Messing 3}) deiner aktuellen Karrierestufe unter 'Status' ein.
%     \item Trage deine Klasse (z.B. \textit{Akademiker}) deiner aktuellen Karrierestufe unter 'Klasse' ein.
%   \end{itemize}
% \end{DndReadAloud}

\subsection{3. Spielwerte}
Bestimme deine 10 Attribute:
% \begin{DndReadAloud}
\begin{itemize}[noitemsep]
  \item Würfle 12 Mal 2W10 und notiere dir die Ergebnisse. Damit erhältst du 12 Werte zwischen 2 und 20.
  \item Streiche das höchste und niedrigste Ergebnis.
  \item Verteile die restlichen 10 Werte auf deine Spielwerte in der Reihe \textit{Anfagngswert}.
  \item Addiere die Basiswerte für dein Volk aus der Tabelle \textit{Basiswerte} auf den \textit{Anfagngswert} auf.
  \item Verteile insgesamt 5 Steigerungen in der Reihe \textit{Steigerungen} aus den Attributen deiner Karrierestufe.
  \item Trage dir deine Schicksalspunkte aus der Tabelle \textit{Basiswerte} am Charakterblatt ein. Deine Glückspunkte sind gleich deiner Schicksalspunkte.
\end{itemize}
% \end{DndReadAloud}

\begin{DndTable}[header=Basiswerte]{lXXXXX}
  & \textbf{Mensch} & \textbf{Zwerg} & \textbf{Halbling} \\
  KG               & $20$           & $30$          & $10$                \\
  BF               & $20$           & $20$          & $20$                \\
  ST               & $20$           & $20$          & $10$                \\
  WI               & $20$           & $30$          & $20$                \\
  I                & $20$           & $20$          & $30$                \\
  GW               & $20$           & $10$          & $20$                \\
  GS               & $20$           & $30$          & $30$                \\
  IN               & $20$           & $20$          & $20$                \\
  WK               & $20$           & $30$          & $30$                \\
  CH               & $20$           & $10$          & $30$                \\
  Schicksalspunkte & $3$            & $2$            & $2$
\end{DndTable}
Anmerkung: \textit{BF} und \textit{I} wurden für Halblinge vertauscht.
Für Zwerge wurde der \textit{WK}-Bonus von 40 auf 30 reduziert.

\subsection[]{4. Volks-Talente}
Bestimme die fünf Talente deines Volkes. Diese sind im Grundregelwerk auf Seite 36 gelistet.
Anstatt zufällige Talente zu würfeln, kann der Spieler diese auch selbst aus der Tabelle auswählen.\\
Die Tabelle wird mit den Talenten \textit{Gesellig}, \textit{Beharrlich} und \textit{Genügsam} erweitert.

\begin{DndComment}{Unkenruf}
  Dieses Talent ist mehr Fluff als eigentliches Talent und bietet abgesehen von einem Meta-Vorteil (Charakter stirbt) keine Vorteile beim Spielen.
  Der Spieler sucht anstatt Unkenruf ein anderes Talent aus.
\end{DndComment}

\subsection[]{5. Volks-Fähigkeiten}
Entscheide dich für 6 Fähigkeiten von deinem Volk (GRW Seite 36).
Vergebe für 3 Fähigkeiten 5 Steigerungen, und für die restlichen 3 Fähigkeiten 3 Steigerungen.
Notiere dir \textit{Sprache (Reikspiel)} mit $0$ Steigerungen.

\subsection[]{6. Karriere-Talent}
Wähle ein Talent in deiner Karrierestufe.

\subsection[]{7. Karriere-Fähigkeiten}
Verteile 40 Steigerungen in die Fähigkeiten deiner Karrierestufe.
Pro Fähigkeit dürfen in diesem Schritt maximal 10 Punkte vergeben werden.

\subsection[]{8. Ausrüstung}
Du bekommst die Ausrüstung deiner Klasse (Seite 37) und deiner ersten Karrierestufe.
Die angegebenen Gegestände sind ein Beispiel. Der Spieler ist angehalten seine Ausrüstung (sinnvoll) zu erweitern.\\
Zusätzlich erhält der Spieler ein Startgeld:
\begin{DndTable}[header=Startgeld]{lX}
  \textbf{Rang} & \textbf{Vermögen} \\
  Messing               & 2W10 Messinggroschen pro Ansehens-Stufe                 \\
  Silber               & 1W10 Silberschilling pro Ansehens-Stufe                 \\
  Gold & 1 Goldkrone pro Ansehens-Stufe
\end{DndTable}

\subsection[]{9. Werte}
\begin{itemize}
  \item Fülle die Reihe \textit{Wert} deiner \textit{Attribute} und \textit{Fähigkeiten} aus.
  \item Der Spieler beginnt mit 0 Korrumpierungspunkten.
  \item Berechne deine Lebenspunkte (siehe Seite 33).
\end{itemize}


\subsection[]{10. Details}
Fülle deine Charakterdetails aus:
\begin{itemize}
  \item Alter
  \item Haarfarbe
  \item Augenfarbe
  \item Körpergröße
  \item Bewegung
\end{itemize}

Makiere dir deine \textit{Attribute} und \textit{Fähigkeiten} fürs Steigern.

\end{document}
