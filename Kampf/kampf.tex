\documentclass[a4paper, fontsize=10pt twocolumn]{scrartcl}

\usepackage[margin=20mm]{geometry}
\usepackage{times}
\usepackage[labelformat=empty]{caption} % remove caption
\usepackage{multicol}
\usepackage{tabularx}
\usepackage[table,xcdraw]{xcolor}
\renewcommand{\arraystretch}{1.2}
\newcolumntype{R}{>{\raggedright\arraybackslash}X}%

\setlength{\columnsep}{1cm}

\begin{document}
\section{Heldenpunkte}
\begin{multicols}{2}
    \subsection{Schicksal}
    Verdient durch extrem heldenhaften Taten oder von großer Bedeutung.
    \begin{itemize}
        \item Statt zu sterben geht dein Charakter besinnungslos zu Boden und wird aus dem Geschehen genommen; dein SC überlebt, ganz egal wie extrem die Umstände.
        \item Ignoriere den Schaden einer einzelnen Quelle.
    \end{itemize}

    \subsection{Glück}
    Wird am Beginn des Spieleabends auf die Anzhal der Schicksalpunkte gesetzt.
    \begin{itemize}
        \item Wiederhole einen misslungenen Wurf.
        \item Addiere bei einem Wurf nach dem Würfeln $+1$ EG.
        \item Entscheide am Anfang der Runde, wann du in der Runde am Zug sein möchtest, unabhängig von der Initiative-Reihenfolge.
    \end{itemize}

    \vfill\null
    \columnbreak

    \subsection{Zähigkeit}
    Verdient wenn man etwas im Sinne deiner Motivation extrem Bedeutsames gemacht hast.
    \begin{itemize}
        \item Du entscheidest dich, nicht für eine Mutation zu würfeln. Da du nicht mutierst, verlierst du auch keine Korrumpierungspunkte.
        \item  Statt für einen Wurf zu würfeln, wählst du dein Wurfergebnis aus. Bei Vergleichenden Würfen gewinnst du stets mit mindestens +1 EG. Du kannst dies auch wählen, wenn du den Wurf bereits durchgeführt hast und gescheitert bist.
    \end{itemize}

    \subsection{Mut}
    Mutpunkte erhältst du zurück, wann immer du dich gemäß deiner Motivation verhältst.
    \begin{itemize}
        \item Werde bis zum Ende der nächsten Runde immun gegen Psychologie.
        \item Ignoriere bis zum Beginn der nächsten Runde alle Modifikatoren für eine Kritische Verletzung.
        \item 1 Zustand entfernen; wenn du auf diese Weise den Zustand Niedergestreckt beseitigst, erhältst du zusätzlich 1 LP zurück, während du zurück auf die Beine kommst.
    \end{itemize}

\end{multicols}

\section{Kampf}
\begin{table}[!ht]
    \centering
    \begin{tabularx}{\textwidth}{l|l|l}
        \multicolumn{3}{c}{\cellcolor{gray!25} \textbf{Fernkampf Reichweite}} \\ \cline{1-3}
                           & Reichweite & Modifikator                         \\ \cline{1-3}
        Kernschussweite    & $1/10$     & +40                                 \\
        Kurze Reichweite   & 1/2        & +20                                 \\
        Standard           & 1          & 0                                   \\
        Lange Reichweite   & 2          & -10                                 \\
        Extreme Reichweite & 3          & -30
    \end{tabularx}
\end{table}


\begin{table}[!ht]
    \centering
    \begin{tabularx}{\textwidth}{l|l}
        \multicolumn{2}{c}{\cellcolor{gray!25} \textbf{Treffertabelle}} \\ \cline{1-2}
        Wurf  & Zone                                                    \\ \cline{1-2}
        01-09 & Kopf                                                    \\
        10-24 & Linker Arm                                              \\
        24-44 & Rechter Arm                                             \\
        45-79 & Körper                                                  \\
        80-89 & Linkes Bein                                             \\
        90-00 & Rechtes Bein
    \end{tabularx}
\end{table}

\clearpage
\section{Reisen}
\begin{table}[!ht]
    \centering
    \begin{tabularx}{\textwidth}{l|l|l|l}
        \multicolumn{4}{c}{\cellcolor{gray!25} \textbf{Reisekosten}} \\ \cline{1-4}
        Transport & Bewegung & Kosten & Distanz                      \\ \cline{1-4}
        Kutsche   & 6        &        &                              \\
        -Innen    &          & 2G     & pro 1.6 km                   \\
        -Außen    &          & 1G     & pro 1.6 km                   \\
        Barke     & 8        &        &                              \\
        -Kabine   &          & 5G     & pro 1.6 km                   \\
        -Deck     &          & 2G     & pro 1.6 km                   \\
        Droschke  & 6        & 3G     & pro 250m                     \\
        Fähre     & 4        & 1G     & pro 20m
    \end{tabularx}
\end{table}

\begin{table}[!ht]
    \centering
    \begin{tabularx}{\textwidth}{l|l|l|l}
        \multicolumn{4}{c}{\cellcolor{gray!25} \textbf{Verfügbarkeit}} \\ \cline{1-4}
                   & Dorf  & Kleinstadt & Stadt                        \\ \cline{1-4}
        Verbreitet & 100\% & 100\%      & 100\%                        \\
        Knapp      & 30\%  & 60\%       & 90\%                         \\
        Selten     & 15\%  & 30\%       & 45\%                         \\
        Exotisch   & -     & -          & -
    \end{tabularx}
\end{table}

\end{document}